\documentclass[mathserif]{beamer}%
%
\xdef\sharedDir{../_shared_}%
\RequirePackage{\sharedDir/styles/slides}%
%
\subtitle{5. Stochastic Hill Climbing}%
%
\begin{document}%
\startPresentation%
%
\section{Introduction}%
%
\begin{frame}%
\frametitle{Information from Good Solutions}%
\begin{itemize}%
\item Our first algorithm, random sampling, was not very efficient.
\item<2-> It does not make any use of the information it ``sees'' during the optimization process.%
\item<3-> Each search step consists of creating an entirely new, entirely random candidate solution.%
\item<4-> Each search step is thus independent of all prior steps.%
\item<5-> \alert{Is this a good idea?}%
\item<6-> Probably not.%
\item<7-> In almost all practical scenarios, good solutions are somewhat similar to other good solutions.%
\item<8-> In other words, every good solution we see is some useful information.%%
\end{itemize}%
\end{frame}%
%
\begin{frame}[t]%
\frametitle{Basic Idea}%
\begin{itemize}%
\item So how we can make use of the information we have seen during the search?%
\item<3-> Instead of generating a completely random new candidate solution in each step\dots%
\item<4-> {\dots}why can't we try to iteratively improve the best solution we have seen so far?%
\end{itemize}%
\end{frame}%
%
\section{Algorithm Concept}%
%
\begin{frame}[t]%
\frametitle{Stochastic Hill Climbing}%
\begin{itemize}%
\item This is the concept of \alert{Local Search}\cite{WGOEB,HS2005SLSFAA,RN2002AI,S2003ITSSAO} and its simplest realization is \alert{Stochastic Hill Climbing}\cite{WGOEB}.%
\item<2-> Simple Concept\uncover<3->{:%
\begin{enumerate}%
\item create random initial solution%
\item<4-> make a modified copy of best-so-far solution%
\item<5-> if it is better, it becomes the new best-so-far solution (if it is not better, discard it).%
\item<6-> go back to \enumerateItem{2} (until the time is up)%
\end{enumerate}}%
\end{itemize}%
\locateGraphic{2}{width=1.00001\paperwidth}{graphics/hill_climbing/hill_climbing}{-0.000005}{0.382}%
\end{frame}%
%
\begin{frame}%
\frametitle{Causality}%
\begin{itemize}%
\item Local searches like hill climbers exploit a property of many optimization problems called \alert{causality}\cite{R1973ES,R1994ES,WCT2012EOPABT,WZCN2009WIOD}.%
\item<2-> Causality means that small changes in the features of an object (or candidate solution) also lead to small changes in its behavior (or objective value).%
\item<3-> If an optimization problem exhibits causality, then there should be good solutions that are similar to other good solutions.%
\item<4-> The idea is that if we have a good candidate solution, then there may exist similar solutions which are better.%
\item<5-> We hope to find one of them and then continue trying to do the same from there.%
\end{itemize}%
\end{frame}%
%
\begin{frame}%
\frametitle{Implementation of the Stochastic Hill Climber}%
\listing{0.9}{1.3}{language=myJava,mathescape=false}{code/HillClimber.java}%
\end{frame}%
%
\endPresentation%
\end{document}%%
\endinput%
%
