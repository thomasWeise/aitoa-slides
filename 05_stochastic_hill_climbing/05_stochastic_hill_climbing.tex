\documentclass[mathserif]{beamer}%
%
\xdef\sharedDir{../_shared_}%
\RequirePackage{\sharedDir/styles/slides}%
%
\subtitle{5. Stochastic Hill Climbing}%
%
\begin{document}%
\startPresentation%
%
\section{Introduction}%
%
\begin{frame}%
\frametitle{Information from Good Solutions}%
\begin{itemize}%
\item Our first algorithm, random sampling, was not very efficient.
\item<2-> It does not make any use of the information it ``sees'' during the optimization process.%
\item<3-> Each search step consists of creating an entirely new, entirely random candidate solution.%
\item<4-> Each search step is thus independent of all prior steps.%
\item<5-> \alert{Is this a good idea?}%
\item<6-> Probably not.%
\item<7-> In almost all practical scenarios, good solutions are somewhat similar to other good solutions.%
\item<8-> In other words, every good solution we see is some useful information.%%
\end{itemize}%
\end{frame}%
%
\begin{frame}[t]%
\frametitle{Basic Idea}%
\begin{itemize}%
\item So how we can make use of the information we have seen during the search?%
\item<3-> Instead of generating a completely random new candidate solution in each step\dots%
\item<4-> {\dots}why can't we try to iteratively improve the best solution we have seen so far?%
\end{itemize}%
\end{frame}%
%
\section{Algorithm Concept}%
%
\begin{frame}[t]%
\frametitle{Stochastic Hill Climbing}%
\begin{itemize}%
\item This is the concept of \alert{Local Search}\cite{WGOEB,HS2005SLSFAA,RN2002AI,S2003ITSSAO} and its simplest realization is \alert{Stochastic Hill Climbing}\cite{WGOEB}.%
\item<2-> Simple Concept\uncover<3->{:%
\begin{enumerate}%
\item create random initial solution%
\item<4-> make a modified copy of best-so-far solution%
\item<5-> if it is better, it becomes the new best-so-far solution (if it is not better, discard it).%
\item<6-> go back to \enumerateItem{2} (until the time is up)%
\end{enumerate}}%
\end{itemize}%
\locateGraphic{2}{width=1.00001\paperwidth}{graphics/hill_climbing/hill_climbing}{-0.000005}{0.382}%
\end{frame}%
%
\begin{frame}%
\frametitle{Implementation of the Stochastic Hill Climber}%
\only<1>{\listing{0.9}{1.3}{language=myJava}{code/HillClimber_01.java}}%
\only<2>{\listing{0.9}{1.3}{language=myJava}{code/HillClimber_02.java}}%
\only<3>{\listing{0.9}{1.3}{language=myJava}{code/HillClimber_03.java}}%
\only<4>{\listing{0.9}{1.3}{language=myJava}{code/HillClimber_04.java}}%
\only<5>{\listing{0.9}{1.3}{language=myJava}{code/HillClimber_05.java}}%
\only<6>{\listing{0.9}{1.3}{language=myJava}{code/HillClimber_06.java}}%
\only<7>{\listing{0.9}{1.3}{language=myJava}{code/HillClimber_07.java}}%
\only<8>{\listing{0.9}{1.3}{language=myJava}{code/HillClimber_08.java}}%
\only<9>{\listing{0.9}{1.3}{language=myJava}{code/HillClimber_09.java}}%
\only<10>{\listing{0.9}{1.3}{language=myJava}{code/HillClimber_10.java}}%
\only<11>{\listing{0.9}{1.3}{language=myJava}{code/HillClimber_11.java}}%
\only<12>{\listing{0.9}{1.3}{language=myJava}{code/HillClimber_12.java}}%
\only<13>{\listing{0.9}{1.3}{language=myJava}{code/HillClimber_13.java}}%
\only<14>{\listing{0.9}{1.3}{language=myJava}{code/HillClimber_14.java}}%
\only<15>{\listing{0.9}{1.3}{language=myJava}{code/HillClimber.java}}%
\end{frame}%
%
\begin{frame}%
\frametitle{Causality}%
\begin{itemize}%
\item Local searches like hill climbers exploit a property of many optimization problems called \alert{causality}\cite{R1973ES,R1994ES,WCT2012EOPABT,WZCN2009WIOD}.%
\item<2-> Causality means that small changes in the features of an object (or candidate solution) also lead to small changes in its behavior (or objective value).%
\item<3-> If an optimization problem exhibits causality, then there should be good solutions that are similar to other good solutions.%
\item<4-> The idea is that if we have a good candidate solution, then there may exist similar solutions which are better.%
\item<5-> We hope to find one of them and then continue trying to do the same from there.%
\end{itemize}%
\end{frame}%
%
\section{Ingredient: Unary Search Operator}%
%
\begin{frame}[t]%
\frametitle{Unary Search Operator}%
\begin{itemize}%
\item Our hill climber must be able to make modified copies of an existing point~$\sespel\in\searchSpace$ in order to find these better points.%
\item<2-> A unary search operator accepts on existing point~$\sespel\in\searchSpace$ and creates a modified copy of it.%
\item<3-> It must make sure that the modified copy is still a valid element of~$\searchSpace$.%
\item<4-> It should ideally be randomized, i.e., applying it twice to the same point~$\sespel$ should yield different results.%
\only<6->{%
\item<6-> How can we implement this for our JSSP scenario?%
\item<7-> \alert{Easy: Just swap two (different) job IDs in the string!}%
\item<8-> Since the numbers of occurrences of the IDs will not change, the new strings will be valid.%
}%
\end{itemize}%
\only<5>{\listing{0.9}{0.95}{language=myJava}{../02_structure/code/IUnarySearchOperator.java}}%
\end{frame}%
%
\begin{frame}%
\frametitle{Example for our \algoStyle{1swap} Operator}%
\locateGraphic{1}{width=0.95\paperwidth}{graphics/unary_1swap_demo/unary_1swap_demo_1}{0.025}{0.3}%
\locateGraphic{2}{width=0.95\paperwidth}{graphics/unary_1swap_demo/unary_1swap_demo_2}{0.025}{0.3}%
\locateGraphic{3}{width=0.95\paperwidth}{graphics/unary_1swap_demo/unary_1swap_demo_3}{0.025}{0.3}%
\locateGraphic{4}{width=0.95\paperwidth}{graphics/unary_1swap_demo/unary_1swap_demo_4}{0.025}{0.3}%
\locateGraphic{5}{width=0.95\paperwidth}{graphics/unary_1swap_demo/unary_1swap_demo_5}{0.025}{0.3}%
\locateGraphic{6}{width=0.95\paperwidth}{graphics/unary_1swap_demo/unary_1swap_demo_6}{0.025}{0.3}%
\locateGraphic{7}{width=0.95\paperwidth}{graphics/unary_1swap_demo/unary_1swap_demo_7}{0.025}{0.3}%
\locateGraphic{8}{width=0.95\paperwidth}{graphics/unary_1swap_demo/unary_1swap_demo_8}{0.025}{0.3}%
\end{frame}%
%
\begin{frame}%
\only<1>{\listing{0.9}{1.05}{language=myJava}{code/JSSPUnaryOperator1Swap_1.java}}%
\only<2>{\listing{0.9}{1.05}{language=myJava}{code/JSSPUnaryOperator1Swap_2.java}}%
\only<3>{\listing{0.9}{1.05}{language=myJava}{code/JSSPUnaryOperator1Swap_3.java}}%
\only<4>{\listing{0.9}{1.05}{language=myJava}{code/JSSPUnaryOperator1Swap_4.java}}%
\only<5>{\listing{0.9}{1.05}{language=myJava}{code/JSSPUnaryOperator1Swap_5.java}}%
\only<6>{\listing{0.9}{1.05}{language=myJava}{code/JSSPUnaryOperator1Swap_6.java}}%
\only<7>{\listing{0.9}{1.05}{language=myJava}{code/JSSPUnaryOperator1Swap_7.java}}%
\only<8>{\listing{0.9}{1.05}{language=myJava}{code/JSSPUnaryOperator1Swap_8.java}}%
\only<9>{\listing{0.9}{1.05}{language=myJava}{code/JSSPUnaryOperator1Swap.java}}%
\end{frame}%
%
\section{Experiment and Analysis}%
%
\begin{frame}[t]%
\frametitle{So what do we get?}%
%
\only<3->{%
\begin{center}%
\only<3,5,7,9>{\small{\algoStyle{rs}: median result of 3~min of random sampling}}%
\only<4,6,8,10>{\small{\algoStyle{hc_1swap}: median result of 3~min of hill climber}}%
\end{center}%
}%
%
\only<-2>{%
\begin{itemize}%
\item I execute the program 101~times for each of the instances \instStyle{abz7}, \instStyle{la24}, \instStyle{swv15}, and \instStyle{yn4}%
\end{itemize}%
\uncover<2->{%
\begin{center}%
\resizebox{0.9\linewidth}{!}{%
\begin{tabular}{|l|l|r|r|r|r|r|r|}%
\hline%
&&\multicolumn{4}{c|}{\textbf{makespan}}&\multicolumn{2}{c|}{\textbf{last improvement}}\\%
\hline%
$\instance$&algo&best&mean&med&sd&med(t)&med(FEs)\\%
\hline%
\hline%
\instStyle{abz7}&\algoStyle{rs}&895&947&949&\textcolor{green}{\textbf{12}}&85s&6'512'505\\%
&\algoStyle{hc_1swap}&\textcolor{green}{\textbf{717}}&\textcolor{green}{\textbf{800}}&\textcolor{green}{\textbf{798}}&28&0s&16'978\\%
\hline%
\instStyle{la24}&\algoStyle{rs}&1153&1206&1208&\textcolor{green}{\textbf{15}}&82s&15'902'911\\%
&\algoStyle{hc_1swap}&\textcolor{green}{\textbf{999}}&\textcolor{green}{\textbf{1095}}&\textcolor{green}{\textbf{1086}}&56&0s&6'612\\%
\hline%
\instStyle{swv15}&\algoStyle{rs}&4988&5166&5172&\textcolor{green}{\textbf{50}}&87s&5'559'124\\
&\algoStyle{hc_1swap}&\textcolor{green}{\textbf{3837}}&\textcolor{green}{\textbf{4108}}&\textcolor{green}{\textbf{4108}}&137&1s&104'598\\%
\hline%
\instStyle{yn4}&\algoStyle{rs}&1460&1498&1499&\textcolor{green}{\textbf{15}}&76s&4'814'914\\%
&\algoStyle{hc_1swap}&\textcolor{green}{\textbf{1109}}&\textcolor{green}{\textbf{1222}}&\textcolor{green}{\textbf{1220}}&48&0s&31'789\\%
%
\hline%
\end{tabular}%
}%
\end{center}%
}%
}%
%
\locateGraphic{3}{width=0.95\paperwidth}{graphics/rs_gantt/gantt_rs_abz7_949}{0.025}{0.23}%
\locateGraphic{4}{width=0.95\paperwidth}{graphics/hc_1swap_gantt/gantt_hc_1swap_abz7_949}{0.025}{0.23}%
\locateGraphic{5}{width=0.95\paperwidth}{graphics/rs_gantt/gantt_rs_la24_1208}{0.025}{0.23}%
\locateGraphic{6}{width=0.95\paperwidth}{graphics/hc_1swap_gantt/gantt_hc_1swap_la24_1208}{0.025}{0.23}%
\locateGraphic{7}{width=0.95\paperwidth}{graphics/rs_gantt/gantt_rs_swv15_5172}{0.025}{0.23}%
\locateGraphic{8}{width=0.95\paperwidth}{graphics/hc_1swap_gantt/gantt_hc_1swap_swv15_5172}{0.025}{0.23}%
\locateGraphic{9}{width=0.95\paperwidth}{graphics/rs_gantt/gantt_rs_yn4_1499}{0.025}{0.23}%
\locateGraphic{10}{width=0.95\paperwidth}{graphics/hc_1swap_gantt/gantt_hc_1swap_yn4_1499}{0.025}{0.23}%
\end{frame}%
%
\begin{frame}[t]%
\frametitle{Progress over Time}%
\locateGraphic{2-3}{width=0.9\paperwidth}{graphics/hc_1swap_progress/hc_1swap_progress_abz7_log}{0.05}{0.2}%
\locateGraphic{4}{width=0.9\paperwidth}{graphics/hc_1swap_progress/hc_1swap_progress_la24_log}{0.05}{0.2}%
\locateGraphic{5}{width=0.9\paperwidth}{graphics/hc_1swap_progress/hc_1swap_progress_swv15_log}{0.05}{0.2}%
\locateGraphic{6-7}{width=0.9\paperwidth}{graphics/hc_1swap_progress/hc_1swap_progress_yn4_log}{0.05}{0.2}%
\begin{center}%
What progress does the algorithm make over time?
\par%
\vspace{0.65\paperheight}%
\uncover<3->{First we have much progress\dots\par\uncover<7->{{\dots}\alert{but then the hill climber stagnates!}}}%
\end{center}
\end{frame}%
%
\begin{frame}[t]%
\frametitle{But we waste time\dots}%
\locateGraphic{2-3}{width=0.9\paperwidth}{graphics/hc_1swap_progress/hc_1swap_progress_abz7}{0.05}{0.2}%
\locateGraphic{4}{width=0.9\paperwidth}{graphics/hc_1swap_progress/hc_1swap_progress_la24}{0.05}{0.2}%
\locateGraphic{5}{width=0.9\paperwidth}{graphics/hc_1swap_progress/hc_1swap_progress_swv15}{0.05}{0.2}%
\locateGraphic{6}{width=0.9\paperwidth}{graphics/hc_1swap_progress/hc_1swap_progress_yn4}{0.05}{0.2}%
\begin{center}%
What if we look at this without log-scaling the time axis?%%
\par%
\vspace{0.65\paperheight}%
\uncover<3->{\alert{Then it looks even much worse!}}%
\end{center}%
\end{frame}%
%
\begin{frame}%
\frametitle{Indeed, we waste time!}%
\begin{center}%
\resizebox{0.9\linewidth}{!}{%
\begin{tabular}{|l|l|r|r|r|r|r|r|}%
\hline%
&&\multicolumn{4}{c|}{\textbf{makespan}}&\multicolumn{2}{c|}{\textbf{last improvement}}\\%
\hline%
$\instance$&algo&best&mean&med&sd&\textcolor<2->{red}{med(t)}&med(FEs)\\%
\hline%
\hline%
\instStyle{abz7}&\algoStyle{rs}&895&947&949&\textcolor{green}{\textbf{12}}&85s&6'512'505\\%
&\algoStyle{hc_1swap}&\textcolor{green}{\textbf{717}}&\textcolor{green}{\textbf{800}}&\textcolor{green}{\textbf{798}}&28&\textcolor<2->{red}{0s}&16'978\\%
\hline%
\instStyle{la24}&\algoStyle{rs}&1153&1206&1208&\textcolor{green}{\textbf{15}}&82s&15'902'911\\%
&\algoStyle{hc_1swap}&\textcolor{green}{\textbf{999}}&\textcolor{green}{\textbf{1095}}&\textcolor{green}{\textbf{1086}}&56&\textcolor<2->{red}{0s}&6'612\\%
\hline%
\instStyle{swv15}&\algoStyle{rs}&4988&5166&5172&\textcolor{green}{\textbf{50}}&87s&5'559'124\\
&\algoStyle{hc_1swap}&\textcolor{green}{\textbf{3837}}&\textcolor{green}{\textbf{4108}}&\textcolor{green}{\textbf{4108}}&137&\textcolor<2->{red}{1s}&104'598\\%
\hline%
\instStyle{yn4}&\algoStyle{rs}&1460&1498&1499&\textcolor{green}{\textbf{15}}&76s&4'814'914\\%
&\algoStyle{hc_1swap}&\textcolor{green}{\textbf{1109}}&\textcolor{green}{\textbf{1222}}&\textcolor{green}{\textbf{1220}}&48&\textcolor<2->{red}{0s}&31'789\\%
%
\hline%
\end{tabular}%
}%
\end{center}%
\uncover<3->{%
\begin{itemize}%
\item We have three minutes but after about 1~second, our \algoStyle{hc_1swap} algorithm stops improving!%
\end{itemize}%
}%
\end{frame}%
%
\begin{frame}%
\frametitle{Premature Convergence}%
\only<-10>{%
\begin{itemize}%
\item Our algorithm makes most of its progress early during the search.%
\item<2-> Later, it basically stagnates and cannot improve.%
\item<3-> Why is that?%
\item<4-> The search operator \algoStyle{1swap} defines a neighborhood~$N(\sespel)\subset\searchSpace$ around a point~$\sespel$.%
\item<5-> The hill climber can only find solutions which are in the neighborhood of the current best solution.%
\item<6-> Only the schedules that I can reach by swapping two operations of two different jobs.%
\item<7-> Clearly~$|N(\sespel)|\lll|\searchSpace|$!%
\item<8-> What happens if~$\objf(\repMap(\localOptimum{\sespel}))\leq\objf(\repMap(\sespel))\forall \sespel\in N(\localOptimum{\sespel})$ but $\localOptimum{\sespel}$ is not the global optimum?%
\item<9-> Our algorithm gets trapped in the \alert{local optimum} $\localOptimum{\sespel}$ and cannot escape!%
\item<10-> This is called \alert{Premature Convergence}.\cite{WZCN2009WIOD,WCT2012EOPABT}%
\end{itemize}%
}%
\locateGraphic{11}{width=0.75\paperwidth}{graphics/premature_convergence/premature_convergence}{0.125}{0.2}%
\end{frame}%
%
\section{Improved Algorithm Concept 1}%
%
\begin{frame}[t]%
\frametitle{Stochastic Hill Climber with Restarts}%
\begin{itemize}%
\item Idea: We have seen that the results of the hill climber exhibit a relatively \textcolor<1>{blue}{high standard deviation}.%
\item<2-> At the same time, a single \emph{run} of the algorithm \textcolor<2>{red}{converges quickly}.%
\only<3->{%
\item<3-> Let us exploit this variance!%
\item<4-> Idea: If we did not make any progress for a number~$L$ of algorithm steps, we simply restart at a new random solution.%
\item<5-> Of course, we will always remember the overall best solution we ever had (in another variable).%
}%
\end{itemize}%
\only<-2>{%
\begin{center}%
\resizebox{0.9\linewidth}{!}{%
\begin{tabular}{|l|l|r|r|r|r|r|r|}%
\hline%
&&\multicolumn{4}{c|}{\textbf{makespan}}&\multicolumn{2}{c|}{\textbf{last improvement}}\\%
\hline%
$\instance$&algo&best&mean&med&\textcolor{blue}{sd}&\textcolor<2->{red}{med(t)}&med(FEs)\\%
\hline%
\hline%
\instStyle{abz7}&\algoStyle{rs}&895&947&949&12&85s&6'512'505\\%
&\algoStyle{hc_1swap}&717&800&798&\textcolor{blue}{28}&\textcolor<2->{red}{0s}&16'978\\%
\hline%
\instStyle{la24}&\algoStyle{rs}&1153&1206&1208&15&82s&15'902'911\\%
&\algoStyle{hc_1swap}&999&1095&1086&\textcolor{blue}{56}&\textcolor<2->{red}{0s}&6'612\\%
\hline%
\instStyle{swv15}&\algoStyle{rs}&4988&5166&5172&50&87s&5'559'124\\
&\algoStyle{hc_1swap}&3837&4108&4108&\textcolor{blue}{137}&\textcolor<2->{red}{1s}&104'598\\%
\hline%
\instStyle{yn4}&\algoStyle{rs}&1460&1498&1499&15&76s&4'814'914\\%
&\algoStyle{hc_1swap}&1109&1222&1220&\textcolor{blue}{48}&\textcolor<2->{red}{0s}&31'789\\%
%
\hline%
\end{tabular}%
}%
\end{center}%
}%
\end{frame}%
%
\begin{frame}%
\frametitle{Stochastic Hill Climbing Algorithm with Restarts}%
\only<1>{\listing{0.85}{1.4}{language=myJava}{code/HillClimberWithRestarts_01.java}}%
\only<2>{\listing{0.85}{1.4}{language=myJava}{code/HillClimberWithRestarts_02.java}}%
\only<3>{\listing{0.85}{1.4}{language=myJava}{code/HillClimberWithRestarts_03.java}}%
\only<4>{\listing{0.85}{1.4}{language=myJava}{code/HillClimberWithRestarts_04.java}}%
\only<5>{\listing{0.85}{1.4}{language=myJava}{code/HillClimberWithRestarts_05.java}}%
\only<6>{\listing{0.85}{1.4}{language=myJava}{code/HillClimberWithRestarts_06.java}}%
\only<7>{\listing{0.85}{1.4}{language=myJava}{code/HillClimberWithRestarts_07.java}}%
\only<8>{\listing{0.85}{1.4}{language=myJava}{code/HillClimberWithRestarts_08.java}}%
\only<9>{\listing{0.85}{1.4}{language=myJava}{code/HillClimberWithRestarts_09.java}}%
\only<10>{\listing{0.85}{1.4}{language=myJava}{code/HillClimberWithRestarts_10.java}}%
\only<11>{\listing{0.85}{1.4}{language=myJava}{code/HillClimberWithRestarts_11.java}}%
\only<12>{\listing{0.85}{1.4}{language=myJava}{code/HillClimberWithRestarts_12.java}}%
\only<13>{\listing{0.85}{1.4}{language=myJava}{code/HillClimberWithRestarts_13.java}}%
\only<14>{\listing{0.85}{1.4}{language=myJava}{code/HillClimberWithRestarts_14.java}}%
\only<15>{\listing{0.85}{1.4}{language=myJava}{code/HillClimberWithRestarts_15.java}}%
\only<16>{\listing{0.85}{1.4}{language=myJava}{code/HillClimberWithRestarts_16.java}}%
\only<17>{\listing{0.85}{1.4}{language=myJava}{code/HillClimberWithRestarts_17.java}}%
\only<18>{\listing{0.85}{1.4}{language=myJava}{code/HillClimberWithRestarts_18.java}}%
\only<19>{\listing{0.85}{1.4}{language=myJava}{code/HillClimberWithRestarts_19.java}}%
\only<20>{\listing{0.85}{1.4}{language=myJava}{code/HillClimberWithRestarts_20.java}}%
\only<21>{\listing{0.85}{1.4}{language=myJava}{code/HillClimberWithRestarts_21.java}}%
\only<22>{\listing{0.85}{1.4}{language=myJava}{code/HillClimberWithRestarts.java}}%
\end{frame}%
%
\section{Experiment and Analysis}%
%
\begin{frame}%
\frametitle{Configuring the Algorithm: Parameter~$L$}%
\only<-5,7->{%
\begin{itemize}%
\item We now have an algorithm which, in theory, should be able to utilize some of the variance that we observe in the results of \algoStyle{hc_1swap}.%
\item<2-> We got one problem, though \dots\uncover<3->{~{\dots}actually, it is not just one algorithm, it is an algorithm with a \alert{parameter}~$L$:~\algoStyle{hcr_L_1swap}.}%
\item<4-> What do we do with that?%
\item<5-> Let's take a look.%
\item<7-> If we choose~$L$ too small, we will restart the algorithm too early\uncover<8->{, before it even arrives in a local optimum}%
\item<9-> If we choose~$L$ too large, we will restart too late and thus waste time\uncover<10->{, that we could have used for more restarts}%
\item<11-> $L=2^{14}=16'384$ seems to be a reasonable choice.%
\end{itemize}%
}%
%
\locateGraphic{6}{width=0.95\paperwidth}{graphics/hcr_1swap_config/hcr_1swap_med_over_l}{0.025}{0.13}%
%
\end{frame}%
%
%
\begin{frame}[t]%
\frametitle{So what do we get?}%
%
 \only<3->{%
\begin{center}%
\only<3,5,7,9>{\small{\algoStyle{hc_1swap}: median result of 3~min of hill climber}}%
\only<4,6,8,10>{\small{\algoStyle{hcr_16384_1swap}: median result of 3~min of hill climber which restarts after $L=16'384$ search steps without improvement}}%
\end{center}%
}%
%
\only<-2>{%
\begin{itemize}%
\item I execute the program 101~times for each of the instances \instStyle{abz7}, \instStyle{la24}, \instStyle{swv15}, and \instStyle{yn4}%
\end{itemize}%
\uncover<2->{%
\begin{center}%
\resizebox{0.9\linewidth}{!}{%
\begin{tabular}{|l|l|r|r|r|r|r|r|}%
\hline%
&&\multicolumn{4}{c|}{\textbf{makespan}}&\multicolumn{2}{c|}{\textbf{last improvement}}\\%
\hline%
$\instance$&algo&best&mean&med&sd&med(t)&med(FEs)\\%
\hline%
\hline%
\instStyle{abz7}&\algoStyle{rs}&895&947&949&12&85s&6'512'505\\%                           
&\algoStyle{hc_1swap}&717&800&798&28&0s&16'978\\%                       
&\algoStyle{hcr_16384_1swap}&\textcolor{green}{\textbf{714}}&\textcolor{green}{\textbf{732}}&\textcolor{green}{\textbf{733}}&\textcolor{green}{\textbf{6}}&91s&18'423'530\\%
\hline%    
\instStyle{la24}&\algoStyle{rs}&1153&1206&1208&15&82s&15'902'911\\%                       
&\algoStyle{hc_1swap}&999&1095&1086&56&0s&6'612\\%
&\algoStyle{hcr_16384_1swap}&\textcolor{green}{\textbf{953}}&\textcolor{green}{\textbf{976}}&\textcolor{green}{\textbf{976}}&\textcolor{green}{\textbf{7}}&80s&34'437'999\\%
\hline%    
\instStyle{swv15}&\algoStyle{rs}&4988&5166&5172&50&87s&5'559'124\\%                       
&\algoStyle{hc_1swap}&3837&4108&4108&137&1s&104'598\\%                  
&\algoStyle{hcr_16384_1swap}&\textcolor{green}{\textbf{3752}}&\textcolor{green}{\textbf{3859}}&\textcolor{green}{\textbf{3861}}&\textcolor{green}{\textbf{42}}&92s&11'756'497\\%
\hline%
\instStyle{yn4}&\algoStyle{rs}&1460&1498&1499&15&76s&4'814'914\\%                         
&\algoStyle{hc_1swap}&1109&1222&1220&48&0s&31'789\\%                    
&\algoStyle{hcr_16384_1swap}&\textcolor{green}{\textbf{1081}}&\textcolor{green}{\textbf{1115}}&\textcolor{green}{\textbf{1115}}&\textcolor{green}{\textbf{11}}&91s&14'804'358\\%
%
\hline%
\end{tabular}%
}%
\end{center}%
}%
}%
%
\locateGraphic{3}{width=0.95\paperwidth}{graphics/hc_1swap_gantt/gantt_hc_1swap_abz7_798}{0.025}{0.26}%
\locateGraphic{4}{width=0.95\paperwidth}{graphics/hcr_1swap_gantt/gantt_hcr_16384_1swap_abz7_798}{0.025}{0.26}%
\locateGraphic{5}{width=0.95\paperwidth}{graphics/hc_1swap_gantt/gantt_hc_1swap_la24_1086}{0.025}{0.26}%
\locateGraphic{6}{width=0.95\paperwidth}{graphics/hcr_1swap_gantt/gantt_hcr_16384_1swap_la24_1086}{0.025}{0.26}%
\locateGraphic{7}{width=0.95\paperwidth}{graphics/hc_1swap_gantt/gantt_hc_1swap_swv15_4108}{0.025}{0.26}%
\locateGraphic{8}{width=0.95\paperwidth}{graphics/hcr_1swap_gantt/gantt_hcr_16384_1swap_swv15_4108}{0.025}{0.26}%
\locateGraphic{9}{width=0.95\paperwidth}{graphics/hc_1swap_gantt/gantt_hc_1swap_yn4_1220}{0.025}{0.26}%
\locateGraphic{10}{width=0.95\paperwidth}{graphics/hcr_1swap_gantt/gantt_hcr_16384_1swap_yn4_1220}{0.025}{0.26}%
\end{frame}%
%
\begin{frame}[b]%
\frametitle{Progress over Time}%
\locateGraphic{2-4}{width=0.9\paperwidth}{graphics/hcr_1swap_progress/hcr_1swap_progress_abz7_log}{0.05}{0.12}%
\locateGraphic{5}{width=0.9\paperwidth}{graphics/hcr_1swap_progress/hcr_1swap_progress_la24_log}{0.05}{0.12}%
\locateGraphic{6}{width=0.9\paperwidth}{graphics/hcr_1swap_progress/hcr_1swap_progress_swv15_log}{0.05}{0.12}%
\locateGraphic{7}{width=0.9\paperwidth}{graphics/hcr_1swap_progress/hcr_1swap_progress_yn4_log}{0.05}{0.12}%
\locateGraphic{8-9}{width=0.9\paperwidth}{graphics/hcr_1swap_progress/hcr_1swap_progress_abz7}{0.05}{0.12}%
\locateGraphic{10}{width=0.9\paperwidth}{graphics/hcr_1swap_progress/hcr_1swap_progress_la24}{0.05}{0.12}%
\locateGraphic{11}{width=0.9\paperwidth}{graphics/hcr_1swap_progress/hcr_1swap_progress_swv15}{0.05}{0.12}%
\locateGraphic{12}{width=0.9\paperwidth}{graphics/hcr_1swap_progress/hcr_1swap_progress_yn4}{0.05}{0.12}%
\begin{center}%
What progress does the algorithm make over time?
\uncover<3->{%
\begin{itemize}%
\item First it behaves like the normal hill climber%
\item<4-> But it keeps improving.%
\item<9-> Although we still do not use the available time very well\dots%
\end{itemize}%
}%
\end{center}%
\end{frame}%
%
\section{Improved Algorithm Concept 2}%
%
\begin{frame}%
\frametitle{Drawbacks of Restarts}%
\begin{itemize}%
\item A restarted algorithm is still the same algorithm, just restarted.%
\item<2-> If there are many more local optima than global optima, every restart will probably end again in a local optimum.%
\item<3-> If there are many more ``bad'' local optima than ``good'' local optima, every restart will probably end in a ``bad'' local optimum.%
\item<4-> While restarts improve the chance to find better solutions, they cannot solve the intrinsic shortcomings of an algorithm.\medskip%
\item<5-> Another problem is: With every restart we throw away all accumulated knowledge and information of the current run.%
\item<6-> Restarts are therefore also wasteful.%
\end{itemize}%
\end{frame}%
%
\begin{frame}%
\frametitle{How else can we stop premature convergence?}%
\begin{itemize}%
\item Our \algoStyle{hc_1swap} hill climber will stop improving if it can no longer finder better solutions.%
\item<2-> This happens when it reaches a local optimum.%
\item<3-> A local optimum is a \textcolor<6>{blue}{point~$\localOptimum{\sespel}$} in~$\searchSpace$ where no \textcolor<5>{red}{\algoStyle{1swap}-move} can yield any improvement.%
\item<4-> It does not matter which \textcolor<5>{red}{two job ids I exchange} in the current best \textcolor<6>{blue}{string~$\localOptimum{\sespel}$}, the result is not better than~\textcolor<6>{blue}{$\localOptimum{\sespel}$}.%
\item<5-> Notice: Whether or not a point~$\sespel$ is a local optimum, is determined entirely by the \textcolor<5>{red}{unary search operator}!%
\item<6-> If we had a different operator with a bigger neighborhood, then maybe~\textcolor{blue}{$\localOptimum{\sespel}$} would no longer be a local optimum and we could still improve the results after reaching it\dots%
\end{itemize}%
\end{frame}%
%
\begin{frame}%
\frametitle{Making the neighborhood bigger}%
\begin{itemize}%
\item Two solutions~$\sespel_1$ and~$\sespel_2$ are ``neighbors'' if I can reach~$\sespel_2$ by applying the search operator one time to~$\sespel_1$.%
\item<2-> The search operator determines which solutions are ``neighbors''.%
\item<3-> The neighborhood determines what a local optimum is.%
\item<4-> Let's make it bigger.%
\item<5-> It always helps to think about the extreme cases first.%
\item<6-> On one hand, we already have \algoStyle{1swap}, which swaps two jobs.\uncover<7->{ This is the smallest step I can imagine.}%
\item<8-> On the other end of the spectrum, we could simply swap all jobs in our points~$\sespel$ randomly.\uncover<9->{ Is this a good idea?\uncover<10->{Probably not: It would turn our algorithm into random sampling!}}%
\item<11-> We should respect the causality: small changes to the solution cause small changes in the objective value -- big changes will lead to unpredictable results.%
\end{itemize}%
\end{frame}%
%
\begin{frame}%
\frametitle{Making the neighborhood bigger}%
\begin{itemize}%
\item Idea: Let's most often swap 2 jobs\uncover<2->{, but sometimes 3\uncover<3->{, less often 4\uncover<4->{, from time to time 5\uncover<5->{, rarely 6\uncover<6->{, hardly ever 7\uncover<7->{, \dots}}}}}}%
\item<8-> \algoStyle{nswap} operator idea\uncover<9->{:%
\begin{enumerate}%
\item flip a coin: if it is heads (\textcolor{red}{50\% probability}), we will swap \textcolor{red}{2} job ids (and stop).%
\item<10-> otherwise (it was tail), we again flip a coin.\uncover<11->{ if it is heads (50\% probability, now \textcolor{blue}{25\%} in total), we will swap \textcolor{blue}{3} job ids (and stop).}%
\item<12-> otherwise (it was tail), we again flip a coin.\uncover<13->{ if it is heads (50\% probability, now \textcolor{green}{12.5\%} in total), we will swap \textcolor{green}{4} job ids (and stop).}%
\item<14-> otherwise (it was tail), we again flip a coin.\uncover<15->{ if it is heads (50\% probability, now \textcolor{orange}{6.25\%} in total), we will swap \textcolor{orange}{5} job ids (and stop).}%
\item<16-> otherwise (it was tail), we again flip a coin.\uncover<17->{ if it is heads (50\% probability, now \textcolor{cyan}{3.125\%} in total), we will swap \textcolor{cyan}{6} job ids (and stop).}%
\item<18-> and so on.
\end{enumerate}%
}%
\item<19-> We most often make small moves, but sometimes bigger ones.%
\item<20-> Theoretically, we could always escape from any local optimum\only<20>{.}\uncover<21->{, but the probability may sometimes be very very small.}%
\end{itemize}%
\end{frame}%
%
\begin{frame}%
\frametitle{Implementation of the \algoStyle{nswap} Operator}%
\only<1>{\listing{0.9}{1.3}{language=myJava}{code/JSSPUnaryOperatorNSwap_01.java}}%
\only<2>{\listing{0.9}{1.3}{language=myJava}{code/JSSPUnaryOperatorNSwap_02.java}}%
\only<3>{\listing{0.9}{1.3}{language=myJava}{code/JSSPUnaryOperatorNSwap_03.java}}%
\only<4>{\listing{0.9}{1.3}{language=myJava}{code/JSSPUnaryOperatorNSwap_04.java}}%
\only<5>{\listing{0.9}{1.3}{language=myJava}{code/JSSPUnaryOperatorNSwap_05.java}}%
\only<6>{\listing{0.9}{1.3}{language=myJava}{code/JSSPUnaryOperatorNSwap_06.java}}%
\only<7>{\listing{0.9}{1.3}{language=myJava}{code/JSSPUnaryOperatorNSwap_07.java}}%
\only<8>{\listing{0.9}{1.3}{language=myJava}{code/JSSPUnaryOperatorNSwap_08.java}}%
\only<9>{\listing{0.9}{1.3}{language=myJava}{code/JSSPUnaryOperatorNSwap_09.java}}%
\only<10>{\listing{0.9}{1.3}{language=myJava}{code/JSSPUnaryOperatorNSwap_10.java}}%
\only<11>{\listing{0.9}{1.3}{language=myJava}{code/JSSPUnaryOperatorNSwap_11.java}}%
\only<12>{\listing{0.9}{1.3}{language=myJava}{code/JSSPUnaryOperatorNSwap_12.java}}%
\only<13>{\listing{0.9}{1.3}{language=myJava}{code/JSSPUnaryOperatorNSwap_13.java}}%
\only<14>{\listing{0.9}{1.3}{language=myJava}{code/JSSPUnaryOperatorNSwap_14.java}}%
\only<15>{\listing{0.9}{1.3}{language=myJava}{code/JSSPUnaryOperatorNSwap_15.java}}%
\only<16>{\listing{0.9}{1.3}{language=myJava}{code/JSSPUnaryOperatorNSwap_16.java}}%
\only<17>{\listing{0.9}{1.3}{language=myJava}{code/JSSPUnaryOperatorNSwap_17.java}}%
\only<18>{\listing{0.9}{1.3}{language=myJava}{code/JSSPUnaryOperatorNSwap_18.java}}%
\only<19>{\listing{0.9}{1.3}{language=myJava}{code/JSSPUnaryOperatorNSwap.java}}%
\end{frame}%
%
\section{Experiment and Analysis}%
%
\begin{frame}[t]%
\frametitle{So what do we get?}%
%
\only<3->{%
\begin{center}%
\only<3,5,7,9>{\small{\algoStyle{hc_1swap}: median result of 3~min of hill climber} using \algoStyle{1swap}}%
\only<4,6,8,10>{\small{\algoStyle{hc_nswap}: median result of 3~min of hill climber} using \algoStyle{nswap}}%
\end{center}%
}%
%
\only<-2>{%
\begin{itemize}%
\item I execute the program 101~times for each of the instances \instStyle{abz7}, \instStyle{la24}, \instStyle{swv15}, and \instStyle{yn4}%
\end{itemize}%
\uncover<2->{%
\begin{center}%
\resizebox{0.9\linewidth}{!}{%
\begin{tabular}{|l|l|r|r|r|r|r|r|}%
\hline%
&&\multicolumn{4}{c|}{\textbf{makespan}}&\multicolumn{2}{c|}{\textbf{last improvement}}\\%
\hline%
$\instance$&algo&best&mean&med&sd&med(t)&med(FEs)\\%
\hline%
\hline%
\instStyle{abz7}&\algoStyle{hc_1swap}&717&800&798&28&0s&16'978\\%
&\algoStyle{hcr_16384_1swap}&\textcolor{green}{\textbf{714}}&\textcolor{green}{\textbf{732}}&\textcolor{green}{\textbf{733}}&**6&91s&18'423'530\\%
&\algoStyle{hc_nswap}&724&758&758&17&35s&7'781'762\\%
\hline%
\instStyle{la24}&\algoStyle{hc_1swap}&999&1095&1086&56&0s&6'612\\%
&\algoStyle{hcr_16384_1swap}&953&\textcolor{green}{\textbf{976}}&\textcolor{green}{\textbf{976}}&\textcolor{green}{\textbf{7}}&80s&34'437'999\\%
&\algoStyle{hc_nswap}&\textcolor{green}{\textbf{945}}&1018&1016&29&25s&9'072'935\\%
\hline%
\instStyle{swv15}&\algoStyle{hc_1swap}&3837&4108&4108&137&1s&104'598\\%
&\algoStyle{hcr_16384_1swap}&3752&\textcolor{green}{\textbf{3859}}&\textcolor{green}{\textbf{3861}}&\textcolor{green}{\textbf{42}}&92s&11'756'497\\%
&\algoStyle{hc_nswap}&\textcolor{green}{\textbf{3602}}&3880&3872&112&70s&8'351'112\\%
\hline%
\instStyle{yn4}&\algoStyle{hc_1swap}&1109&1222&1220&48&0s&31'789\\%
&\algoStyle{hcr_16384_1swap}&\textcolor{green}{\textbf{1081}}&\textcolor{green}{\textbf{1115}}&\textcolor{green}{\textbf{1115}}&\textcolor{green}{\textbf{11}}&91s&14'804'358\\%
&\algoStyle{hc_nswap}&1095&1162&1160&34&71s&11'016'757\\%
%
\hline%
\end{tabular}%
}%
\end{center}%
}%
}%
%
\locateGraphic{3}{width=0.95\paperwidth}{graphics/hc_1swap_gantt/gantt_hc_1swap_abz7_798}{0.025}{0.26}%
\locateGraphic{4}{width=0.95\paperwidth}{graphics/hc_nswap_gantt/gantt_hc_nswap_abz7_798}{0.025}{0.26}%
\locateGraphic{5}{width=0.95\paperwidth}{graphics/hc_1swap_gantt/gantt_hc_1swap_la24_1086}{0.025}{0.26}%
\locateGraphic{6}{width=0.95\paperwidth}{graphics/hc_nswap_gantt/gantt_hc_nswap_la24_1086}{0.025}{0.26}%
\locateGraphic{7}{width=0.95\paperwidth}{graphics/hc_1swap_gantt/gantt_hc_1swap_swv15_4108}{0.025}{0.26}%
\locateGraphic{8}{width=0.95\paperwidth}{graphics/hc_nswap_gantt/gantt_hc_nswap_swv15_4108}{0.025}{0.26}%
\locateGraphic{9}{width=0.95\paperwidth}{graphics/hc_1swap_gantt/gantt_hc_1swap_yn4_1220}{0.025}{0.26}%
\locateGraphic{10}{width=0.95\paperwidth}{graphics/hc_nswap_gantt/gantt_hc_nswap_yn4_1220}{0.025}{0.26}%
\end{frame}%
%
\begin{frame}[b]%
\frametitle{Progress over Time}%
\locateGraphic{2-5}{width=0.9\paperwidth}{graphics/hc_nswap_progress/hc_nswap_progress_abz7_log}{0.05}{0.12}%
\locateGraphic{6}{width=0.9\paperwidth}{graphics/hc_nswap_progress/hc_nswap_progress_la24_log}{0.05}{0.12}%
\locateGraphic{7}{width=0.9\paperwidth}{graphics/hc_nswap_progress/hc_nswap_progress_swv15_log}{0.05}{0.12}%
\locateGraphic{8}{width=0.9\paperwidth}{graphics/hc_nswap_progress/hc_nswap_progress_yn4_log}{0.05}{0.12}%
\begin{center}%
What progress does the algorithm make over time?%
\uncover<3->{%
\begin{itemize}%
\only<-3>{\item<-3> \algoStyle{hc_nswap} first behaves like \algoStyle{hc_1swap}, because most of the \algoStyle{nswap} moves are the same as \algoStyle{1swap} moves.}%
\only<4->{\item<4-> The rare larger moves allow it to escape from local optima that would trap \algoStyle{hc_1swap}.}%
\only<5->{\item<5-> The hill climber with restarts seems to improve longer.}%
\end{itemize}%
}%
\end{center}%
\end{frame}%
%
\section{Improved Algorithm Concept 3}%
%
\begin{frame}%
\frametitle{Combining the Two Improvements}%
\begin{itemize}%
\item Now we know two ways to improve the performance of our hill climber\only<-1>{.}\uncover<2->{:%
\begin{enumerate}
\item we can restart it\only<3->{ and}%
\item<3-> we can use a unary operator with larger neighborhood that still mostly makes small steps.%
\end{enumerate}%
}%
\item<4-> It is only natural to try to combine these two improvements.%
\end{itemize}%
\end{frame}%
%
\begin{frame}%
\frametitle{Configuring the Algorithm}%
\only<-3,5->{%
\uncover<2-3,5->{%
\begin{itemize}%
\item The \algoStyle{hc_nswap} improves longer than \algoStyle{hc_1swap}%.
\item<3-> We can expect that the number~$L$ of unsuccessful steps before a restart should be higher now.%
\item<5-> Let's choose $L=65'536$, i.e., \algoStyle{hcr_65536_nswap}.
\end{itemize}%
}%
\begin{center}%
\resizebox{0.9\linewidth}{!}{%
\begin{tabular}{|l|l|r|r|r|r|r|r|}%
\hline%
&&\multicolumn{4}{c|}{\textbf{makespan}}&\multicolumn{2}{c|}{\textbf{last improvement}}\\%
\hline%
$\instance$&algo&best&mean&med&sd&med(t)&med(FEs)\\%
\hline%
\hline%
\instStyle{abz7}&\algoStyle{hc_1swap}&717&800&798&28&\textcolor{red}{0s}&\textcolor{red}{16'978}\\%
&\algoStyle{hc_nswap}&724&758&758&17&\textcolor{red}{35s}&\textcolor{red}{7'781'762}\\%
\hline%
\instStyle{la24}&\algoStyle{hc_1swap}&999&1095&1086&56&\textcolor{red}{0s}&\textcolor{red}{6'612}\\%
&\algoStyle{hc_nswap}&945&1018&1016&29&\textcolor{red}{25s}&\textcolor{red}{9'072'935}\\%
\hline%
\instStyle{swv15}&\algoStyle{hc_1swap}&3837&4108&4108&137&\textcolor{red}{1s}&\textcolor{red}{104'598}\\%
&\algoStyle{hc_nswap}&3602&3880&3872&112&\textcolor{red}{70s}&\textcolor{red}{8'351'112}\\%
\hline%
\instStyle{yn4}&\algoStyle{hc_1swap}&1109&1222&1220&48&\textcolor{red}{0s}&\textcolor{red}{31'789}\\%
&\algoStyle{hc_nswap}&1095&1162&1160&34&\textcolor{red}{71s}&\textcolor{red}{11'016'757}\\%
%
\hline%
\end{tabular}%
}%
\end{center}}%
%
\locateGraphic{4}{width=0.95\paperwidth}{graphics/hcr_nswap_config/hcr_nswap_med_over_l}{0.025}{0.13}%
\end{frame}%
%
\section{Experiment and Analysis}%
%
\begin{frame}[t]%
\frametitle{So what do we get?}%
%
\only<3->{%
\begin{center}%
\only<3,5,7,9>{\small{\algoStyle{hcr_16384_1swap}: median result of 3~min of \algoStyle{hcr_16384_1swap}}}%
\only<4,6,8,10>{\small{\algoStyle{hcr_65536_nswap}: median result of 3~min of \algoStyle{hcr_65536_nswap}}}%
\end{center}%
}%
%
\only<-2>{%
\begin{itemize}%
\item I execute the program 101~times for each of the instances \instStyle{abz7}, \instStyle{la24}, \instStyle{swv15}, and \instStyle{yn4}%
\end{itemize}%
\uncover<2->{%
\begin{center}%
\resizebox{0.9\linewidth}{!}{%
\begin{tabular}{|l|l|r|r|r|r|r|r|}%
\hline%
&&\multicolumn{4}{c|}{\textbf{makespan}}&\multicolumn{2}{c|}{\textbf{last improvement}}\\%
\hline%
$\instance$&algo&best&mean&med&sd&med(t)&med(FEs)\\%
\hline%
\hline%
\instStyle{abz7}&\algoStyle{hcr_16384_1swap}&714&732&733&\textcolor{green}{\textbf{6}}&91s&18'423'530\\%
&\algoStyle{hc_nswap}&724&758&758&17&35s&7'781'762\\%
&\algoStyle{hcr_65536_nswap}&\textcolor{green}{\textbf{712}}&\textcolor{green}{\textbf{731}}&\textcolor{green}{\textbf{732}}&6&96s&21'189'358\\%
\hline%
\instStyle{la24}&\algoStyle{hcr_16384_1swap}&953&976&976&\textcolor{green}{\textbf{7}}&80s&34'437'999\\%
&\algoStyle{hc_nswap}&945&1018&1016&29&25s&9'072'935\\%
&\algoStyle{hcr_65536_nswap}&\textcolor{green}{\textbf{942}}&\textcolor{green}{\textbf{973}}&\textcolor{green}{\textbf{974}}&8&71s&31'466'420\\%
\hline%
\instStyle{swv15}&\algoStyle{hcr_16384_1swap}&3752&3859&3861&42&92s&11'756'497\\%
&\algoStyle{hc_nswap}&\textcolor{green}{\textbf{3602}}&3880&3872&112&70s&8'351'112\\%
&\algoStyle{hcr_65536_nswap}&3740&\textcolor{green}{\textbf{3818}}&\textcolor{green}{\textbf{3826}}&\textcolor{green}{\textbf{35}}&89s&10'783'296\\%
\hline%
\instStyle{yn4}&\algoStyle{hcr_16384_1swap}&1081&1115&1115&\textcolor{green}{\textbf{11}}&91s&14'804'358\\%
&\algoStyle{hc_nswap}&1095&1162&1160&34&71s&11'016'757\\%
&\algoStyle{hcr_65536_nswap}&\textcolor{green}{\textbf{1068}}&\textcolor{green}{\textbf{1109}}&\textcolor{green}{\textbf{1110}}&12&78s&18'756'636\\%
%
\hline%
\end{tabular}%
}%
\end{center}%
}%
}%
%
\locateGraphic{3}{width=0.95\paperwidth}{graphics/hcr_1swap_gantt/gantt_hcr_16384_1swap_abz7_733}{0.025}{0.23}%
\locateGraphic{4}{width=0.95\paperwidth}{graphics/hcr_nswap_gantt/gantt_hcr_65536_nswap_abz7_733}{0.025}{0.23}%
\locateGraphic{5}{width=0.95\paperwidth}{graphics/hcr_1swap_gantt/gantt_hcr_16384_1swap_la24_976}{0.025}{0.23}%
\locateGraphic{6}{width=0.95\paperwidth}{graphics/hcr_nswap_gantt/gantt_hcr_65536_nswap_la24_976}{0.025}{0.23}%
\locateGraphic{7}{width=0.95\paperwidth}{graphics/hcr_1swap_gantt/gantt_hcr_16384_1swap_swv15_3861}{0.025}{0.23}%
\locateGraphic{8}{width=0.95\paperwidth}{graphics/hcr_nswap_gantt/gantt_hcr_65536_nswap_swv15_3861}{0.025}{0.23}%
\locateGraphic{9}{width=0.95\paperwidth}{graphics/hcr_1swap_gantt/gantt_hcr_16384_1swap_yn4_1115}{0.025}{0.23}%
\locateGraphic{10}{width=0.95\paperwidth}{graphics/hcr_nswap_gantt/gantt_hcr_65536_nswap_yn4_1115}{0.025}{0.23}%
\end{frame}%
%
\begin{frame}[t]%
\frametitle{Progress over Time}%
\locateGraphic{2-3}{width=0.9\paperwidth}{graphics/hcr_nswap_progress/hcr_nswap_progress_abz7_log}{0.05}{0.2}%
\locateGraphic{4}{width=0.9\paperwidth}{graphics/hcr_nswap_progress/hcr_nswap_progress_la24_log}{0.05}{0.2}%
\locateGraphic{5}{width=0.9\paperwidth}{graphics/hcr_nswap_progress/hcr_nswap_progress_swv15_log}{0.05}{0.2}%
\locateGraphic{6-7}{width=0.9\paperwidth}{graphics/hcr_nswap_progress/hcr_nswap_progress_yn4_log}{0.05}{0.2}%
\begin{center}%
What progress does the algorithm make over time?
\par%
\vspace{0.65\paperheight}%
\uncover<3->{\algoStyle{hcr_nswap} tends to be a tiny little bit better than \algoStyle{hcr_1swap} {\dots} but not much}%
\end{center}
\end{frame}%
%
\section{Summary}%
%
\begin{frame}%
\frametitle{Summary}%
\begin{itemize}%
\item We now have learned a second, more efficient metaheuristic optimization algorithm: stochastic hill climber.%
\item<2-> By making use of the best point in the search space we have seen so far and iteratively trying to improve it, we can dramatically improve the results compared to random sampling.%
\item<3-> Like random sampling, we can apply it to all sorts of problems, as long as we provide the basic structural ingredients.%
\item<4-> Hill climbing is a local search and vulnerable to get trapped in local optima.%
\item<5-> We can try to work around that by implementing good search operators and by restarting the algorithm.%
\end{itemize}%
\end{frame}%
%
\endPresentation%
\end{document}%%
\endinput%
%
