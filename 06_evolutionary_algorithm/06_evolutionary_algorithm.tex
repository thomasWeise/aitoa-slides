\documentclass[mathserif]{beamer}%
%
\xdef\sharedDir{../_shared_}%
\RequirePackage{\sharedDir/styles/slides}%
%
\subtitle{6. Evolutionary Algorithms}%
%
\begin{document}%
\startPresentation%
%
\section{Introduction}%
%
\begin{frame}%
\frametitle{Introduction}%
\begin{itemize}%
\item Hill Climbers are local search.%
\item<2-> They begin at some point~$\sespel$ in the search space and then investigate its neighborhood~$N(\sespel)$.%
\item<3-> The neighborhood is defined by the (unary) search operator, in our case \algoStyle{1swap} or \algoStyle{nswap}.%
\item<4-> If they reach a local optimum~$\localOptimum{\sespel}$, they get trapped.%
\item<5-> We then can restart them, but this means\uncover<5->{
\begin{enumerate}%
\item to start again back at ``zero'' and losing all accumulated information\uncover<6->{ and}%
\item<6-> they may still land again in a local optimum.%
\end{enumerate}%
}%
\item<7-> We can use unary operators which sample non-uniformly from larger neighborhoods, like \algoStyle{nswap}, but the search move needed to escape from a good but non-optimal point might be too unlikely.%
\item<8-> Idea: We could investigate multiple points in the search space at once and use the additional information in a clever way?%
\end{itemize}%
\end{frame}%
%
%
\begin{frame}%
\frametitle{Population-Based Metaheuristics}%
\begin{itemize}%
\item Population-based metaheuristics\cite{WGOEB,H1975GA,BFM1997EA,MF2004HTSIMH,G1989GA} try to maintain a \alert{set} of points in the search space which are iteratively refined.%%
\item<2-> This has a couple of advantages\uncover<3->{:%
\begin{itemize}%
\item We are less likely to get trapped in a single local optimum (because we work on multiple points).%
\item<4-> We might more likely find a better (local) optimum.%
\item<5-> If we have different good points from the search space in our population, we can try to use this additional information\dots%
\end{itemize}%
}%
\end{itemize}%
\end{frame}%
%
\section{Algorithm Concept: Population}%
%
\begin{frame}%
\frametitle{$(\mu+\lambda)$~EA}%
\begin{itemize}%
\item Evolutionary Algorithms (EAs) are the most successful family of population-based metaheuristics.\cite{WGOEB,BFM1997EA,MF2004HTSIMH}%
\item<2-> Here we focus on $(\mu+\lambda)$~EAs\uncover<3->{, which work as follows:%
\begin{enumerate}%
\item Generate a population of $\mu+\lambda$ random points in the search space (and map them to solutions and evaluate them).%
\item<4-> From the population, select the $\mu$~best points as ``parents'' for the next ``generation'' of points, discard the remaining $\lambda$~points.%
\item<5-> Generate $\lambda$~new ``offspring'' points by applying a unary search operator (which creates a randomly modified copy from a selected point).%
\item<6-> Evaluate the $\lambda$~offsprings, add them to the population, and go back to step~\enumerateItem{2}.%
\end{enumerate}%
}%
\end{itemize}%
\end{frame}%
%
\begin{frame}%
\frametitle{Ingredient: Individual Record}%
\listing{0.9}{0.9}{language=myJava,mathescape=false}{code/Individual.java}
\end{frame}%
%
\begin{frame}%
\frametitle{Evolutionary Algorithm Implementation}%
\listing{0.9}{1.4}{language=myJava,mathescape=false}{code/EAnoCR.java}
\end{frame}%
%

\section{Configuring the Algorithm}%
%
\begin{frame}%
\frametitle{Algorithm Parameters}%
\only<-5,7->{%
\begin{itemize}%
\item Our EA has two parameters, $\mu$ and $\lambda$.%
\item<2-> Actually, it has three parameters\only<-2>{.}\uncover<3->{: We can choose \algoStyle{1swap} or \algoStyle{nswap} as unary search operation.}%
\item<4-> For now, let's set $\mu=\lambda$, meaning the number of parents equals the number of offspring in each generation.%
\item<5-> This leaves us two parameters to investigate, so let's take a look.%
\item<7-> Except for \instStyle{swv15}, a setting of $\mu=\lambda=16'384$ seems reasonable.%
\item<8-> Interestingly, there are only little differences between \algoStyle{1swap} and \algoStyle{nswap}\only<-8>{.}\uncover<9->{, but we pick \algoStyle{nswap} because it tends to be the better choice more often.}%
\item<10-> Generally, the EA seems to be quite robust and performs well for many parameter settings (except on \instStyle{swv15}).%
\end{itemize}%
}%
%
\locateGraphic{6}{width=0.95\paperwidth}{graphics/ea_nocr_config/ea_nocr_med_over_mu}{0.025}{0.13}%
%
\end{frame}%
%
\section{Experiment and Analysis}%
%
\begin{frame}[t]%
\frametitle{So what do we get?}%
%
 \only<3->{%
\begin{center}%
\only<3,5,7,9>{\small{\algoStyle{hcr_65536_nswap}: median result of 3~min of the restarted hill climber \algoStyle{hcr_65536_nswap} with $L=65'536$ and \algoStyle{nswap}}}%
\only<4,6,8,10>{\small{\algoStyle{ea_16384_nswap}: median result of 3~min of the EA with $\mu=\lambda=16'384$ with \algoStyle{nswap}}}%
\end{center}%
}%
%
\only<-2>{%
\begin{itemize}%
\item I execute the program 101~times for each of the instances \instStyle{abz7}, \instStyle{la24}, \instStyle{swv15}, and \instStyle{yn4}%
\end{itemize}%
\uncover<2->{%
\begin{center}%
\resizebox{0.9\linewidth}{!}{%
\begin{tabular}{|l|l|r|r|r|r|r|r|}%
\hline%
&&\multicolumn{4}{c|}{\textbf{makespan}}&\multicolumn{2}{c|}{\textbf{last improvement}}\\%
\hline%
$\instance$&algo&best&mean&med&sd&med(t)&med(FEs)\\%
\hline%
\hline%
\instStyle{abz7}&\algoStyle{hcr_65536_nswap}&712&731&732&\textcolor{green}{\textbf{6}}&96s&21'189'358\\%
&\algoStyle{ea_16384_nswap}&\textcolor{green}{\textbf{691}}&\textcolor{green}{\textbf{707}}&\textcolor{green}{\textbf{707}}&8&151s&25'293'859\\%
\hline%
\instStyle{la24}&\algoStyle{hcr_65536_nswap}&\textcolor{green}{\textbf{942}}&973&974&\textcolor{green}{\textbf{8}}&71s&31'466'420\\%
&\algoStyle{ea_16384_nswap}&945&\textcolor{green}{\textbf{968}}&\textcolor{green}{\textbf{967}}&12&39s&10'161'119\\%
\hline%
\instStyle{swv15}&\algoStyle{hcr_65536_nswap}&3740&3818&3826&\textcolor{green}{\textbf{35}}&89s&10'783'296\\%
&\algoStyle{ea_16384_nswap}&\textcolor{green}{\textbf{3577}}&\textcolor{green}{\textbf{3723}}&\textcolor{green}{\textbf{3728}}&50&178s&18'897'833\\%
\hline%
\instStyle{yn4}&\algoStyle{hcr_65536_nswap}&1068&1109&1110&\textcolor{green}{\textbf{12}}&78s&18'756'636\\%
&\algoStyle{ea_16384_nswap}&\textcolor{green}{\textbf{1022}}&\textcolor{green}{\textbf{1063}}&\textcolor{green}{\textbf{1061}}&16&168s&26'699'633\\%
%
\hline%
\end{tabular}%
}%
\end{center}%
}%
}%
%
\locateGraphic{3}{width=0.95\paperwidth}{graphics/hcr_nswap/gantt_hcr_65536_nswap_abz7_732}{0.025}{0.26}%
\locateGraphic{4}{width=0.95\paperwidth}{graphics/ea_nocr_gantt/gantt_ea_nocr_abz7_732}{0.025}{0.26}%
\locateGraphic{5}{width=0.95\paperwidth}{graphics/hcr_nswap/gantt_hcr_65536_nswap_la24_974}{0.025}{0.26}%
\locateGraphic{6}{width=0.95\paperwidth}{graphics/ea_nocr_gantt/gantt_ea_nocr_la24_974}{0.025}{0.26}%
\locateGraphic{7}{width=0.95\paperwidth}{graphics/hcr_nswap/gantt_hcr_65536_nswap_swv15_3826}{0.025}{0.26}%
\locateGraphic{8}{width=0.95\paperwidth}{graphics/ea_nocr_gantt/gantt_ea_nocr_swv15_3826}{0.025}{0.26}%
\locateGraphic{9}{width=0.95\paperwidth}{graphics/hcr_nswap/gantt_hcr_65536_nswap_yn4_1110}{0.025}{0.26}%
\locateGraphic{10}{width=0.95\paperwidth}{graphics/ea_nocr_gantt/gantt_ea_nocr_yn4_1110}{0.025}{0.26}%
\end{frame}%
%
\begin{frame}[t]%
\frametitle{Progress over Time}%
\locateGraphic{2-3}{width=0.9\paperwidth}{graphics/ea_nocr_progress/ea_nocr_progress_abz7_log}{0.05}{0.2}%
\locateGraphic{4}{width=0.9\paperwidth}{graphics/ea_nocr_progress/ea_nocr_progress_la24_log}{0.05}{0.2}%
\locateGraphic{5}{width=0.9\paperwidth}{graphics/ea_nocr_progress/ea_nocr_progress_swv15_log}{0.05}{0.2}%
\locateGraphic{6}{width=0.9\paperwidth}{graphics/ea_nocr_progress/ea_nocr_progress_yn4_log}{0.05}{0.2}%
\locateGraphic{7-8}{width=0.9\paperwidth}{graphics/ea_nocr_progress/ea_nocr_progress_abz7}{0.05}{0.2}%
\locateGraphic{9}{width=0.9\paperwidth}{graphics/ea_nocr_progress/ea_nocr_progress_la24}{0.05}{0.2}%
\locateGraphic{10}{width=0.9\paperwidth}{graphics/ea_nocr_progress/ea_nocr_progress_swv15}{0.05}{0.2}%
\locateGraphic{11}{width=0.9\paperwidth}{graphics/ea_nocr_progress/ea_nocr_progress_yn4}{0.05}{0.2}%
\begin{center}%
What progress does the algorithm make over time?
\par%
\vspace{0.65\paperheight}%
\only<3-6>{On the log-scale, it seems as if the EA first is much slower and very late in the search makes much progress.}%
\only<8->{However, on the linear time scale we can see that it keeps improving slowly but surely during all the time.}%
\end{center}
\end{frame}%
%
\endPresentation%
\end{document}%%
\endinput%
%
