\documentclass[mathserif]{beamer}%
%
\xdef\sharedDir{../_shared_}%
\RequirePackage{\sharedDir/styles/slides}%
%
\subtitle{2. Structure}%
%
\begin{document}%
%
\startPresentation%
%
%
\section{Introduction}%
%
\begin{frame}%
\frametitle{The Structure of Optimization}%
\begin{itemize}%
\item So we know roughly what an optimization problem is and that metaheuristics\cite{aitoa,WGOEB,GK2003HOM,MF2004HTSIMH} are algorithms to solve them.%
\item<2-> But we do not really know yet how that works.%
\item<3-> We will approach this topic based on an example from the field of Smart Manufacturing.
\item<4-> We will first learn about the basic ingredients that make up an optimization task.%
\item<5-> Then we will step-by-step work our way from stupid to good metaheuristics for solving it.%
\end{itemize}%
\end{frame}%
%
\begin{frame}%
\frametitle{Warnings}%
\begin{itemize}%
\item This will be one of the tougher and probably the longest lesson in this lecture.%
\item<2-> We will learn key ideas and concepts that apply to many different scenarios.%
\item<3-> We could look at them from an abstract point of view, similar to an abstract Maths class.%
\item<4-> Then this lesson would be short\dots\uncover<5->{ {\dots}but maybe you won't get a very good feeling for the topic.}%
\item<6-> Instead, we will directly take the abstract concepts and look how they are implemented on one concrete problem.%
\item<7-> This makes the lesson longer, but I hope it will provide for a better understanding.%
\item<8-> The example we will use is \alert{just an example} -- the concepts can be implemented differently for almost all optimization problems.% 
\end{itemize}%
\end{frame}%
%
\begin{frame}[t]%
\frametitle{Components of an Optimization Problem}%
\begin{itemize}%
\item From the perspective of a programmer, we can say that an optimization problem has the following components\uncover<2->{:%
\begin{enumerate}%
\item the input data which specifies the problem instance~$\instance$ to be solved\only<3>{ -- we develop software for solving a class of problems, but this software is applied to specific problem instances, the actual scenarios}%
\item<4-> a data type~$\solutionSpace$ for the candidate solutions~$\solspel\in\solutionSpace$\uncover<2->{, and}%
\item<5-> an objective function~$\objf:\solutionSpace\mapsto\realNumbers$\only<5>{, which rates ``how good'' a candidate solution~$\solspel\in\solutionSpace$ is}.%
\end{enumerate}%
}%
\item<6-> Usually, in order to \alert{practically implement} an optimization approach, there also will be%
\uncover<7->{%
\begin{enumerate}\setcounter{enumi}{3}%
\item a search space~$\searchSpace$\only<7>{, i.e., a simpler data structure for internal use, which can more efficiently be processed by an optimization algorithm than~$\solutionSpace$}\uncover<8->{,}%
\item<8-> a representation mapping~$\repMap:\searchSpace\mapsto\solutionSpace$\only<8>{, which translates ``points''~$\sespel\in\searchSpace$ to candidate solutions~$\solspel\in\solutionSpace$}\uncover<9->{,}%
\item<9-> search operators~$\searchOp:\searchSpace^n\mapsto\searchSpace$\only<9>{, which allow for the iterative exploration of the search space~$\searchSpace$}\uncover<10->{, and}%
\item<10-> a termination criterion\only<10>{, which tells the optimization process when to stop}.%
\end{enumerate}%
}%
\item<11-> Looks complicated\only<12->{, but don't worry}.\only<13->{ We will do this one-by-one}.%
\item<14-> We want to get an understanding of the structure of optimization problems from the metaheuristic perspective by looking at one concrete problem from production planning.%
\end{itemize}%
%
\end{frame}%
%
\section{Example Problem: Job Shop Scheduling}%
%
\begin{frame}%
\frametitle{Job Shop Problem}%
\locateGraphic{1}{width=0.9\paperwidth}{graphics/jssp/jssp_sketch_01}{0.05}{0.2}%
\locateGraphic{2}{width=0.9\paperwidth}{graphics/jssp/jssp_sketch_02}{0.05}{0.2}%
\locateGraphic{3}{width=0.9\paperwidth}{graphics/jssp/jssp_sketch_03}{0.05}{0.2}%
\locateGraphic{4}{width=0.9\paperwidth}{graphics/jssp/jssp_sketch_04}{0.05}{0.2}%
\locateGraphic{5}{width=0.9\paperwidth}{graphics/jssp/jssp_sketch_05}{0.05}{0.2}%
\locateGraphic{6}{width=0.9\paperwidth}{graphics/jssp/jssp_sketch_06}{0.05}{0.2}%
\locateGraphic{7}{width=0.9\paperwidth}{graphics/jssp/jssp_sketch_07}{0.05}{0.2}%
\locateGraphic{8}{width=0.9\paperwidth}{graphics/jssp/jssp_sketch_08}{0.05}{0.2}%
\locateGraphic{9}{width=0.9\paperwidth}{graphics/jssp/jssp_sketch_09}{0.05}{0.2}%
\locateGraphic{10}{width=0.9\paperwidth}{graphics/jssp/jssp_sketch_10}{0.05}{0.2}%
\locateGraphic{11}{width=0.9\paperwidth}{graphics/jssp/jssp_sketch_11}{0.05}{0.2}%
\locateGraphic{12}{width=0.9\paperwidth}{graphics/jssp/jssp_sketch_12}{0.05}{0.2}%
\locateGraphic{13}{width=0.9\paperwidth}{graphics/jssp/jssp_sketch_13}{0.05}{0.2}%
\end{frame}%
%
\begin{frame}%
\frametitle{Job Shop Scheduling Problem}%
\begin{itemize}%
\item The Job Shop Scheduling Problem (JSSP)\cite{GLLRK1979OAAIDSASAS,LLRKS1993SASAAC,L1982RRITTOMS,T199BFBSP,BDP1996TJSSPCANST} is a classical optimization problem.%
\item<2-> We have a factory with~$\jsspMachines$ machines.%
\item<3-> We need to fulfill~$\jsspJobs$ production requests, the \alert{jobs}.%
\item<4-> Each job will need to be processed by some or all of the machines in a job-specific order.%
\item<5-> Also, each job will require a job-specific time at a given machine.%
\item<6-> The goal is to fulfill all tasks as quickly as possible.%
\item<7-> This scenario also encompasses simpler problems, e.g., where all jobs ``are the same.''%
\item<8-> This problem is \mbox{$\NPprefix$-hard}.\cite{K1972RACP,C1971TCOTPP}%
\end{itemize}%
\end{frame}%
%
\begin{frame}%
\frametitle{What we will do}%
\begin{itemize}%
\item In this course, we will use the JSSP as example domain.%
\item<2-> We will discuss all components of an optimization problem based on this example.%
\item<3-> We will discuss several different optimization algorithms -- and apply them to this problem.%
\item<4-> \alert{But}\uncover<5->{~we will do this from an \alert{educational} perspective}%
\item<5-> We will \alert{not} focus on the best possible data structures or highest possible efficiency.%
\item<6-> It needs years of research to get there\dots%
\item<7-> We will, instead, approach the JSSP in the same way you would approach a completely new problem domain\uncover<8->{: develop a working approach\uncover<9->{, test and compare different working approaches\uncover<10->{, (normally you would then improve them further, but we will skip this)}}}%
\end{itemize}%
\end{frame}%
%
\section{Problem Instance}%
%
\begin{frame}%
\frametitle{The Input: Problem Instances}%
\begin{itemize}%
\item The JSSP is a \alert<2>{type} of problem.%
\item<2-> A concrete scenario, with a specific number of machines and with specific jobs, is called an \mbox{\alert<2>{instance~$\jsspInstance$}}.%
\item<3-> It is common in research that there collections of instances for a given problem, so that we can test algorithms and compare their performance (of course, you can only compare results if they are for the same scenario).%
\item<4-> \mbox{Beasley\cite{B1990OLDTPBEM}} manages the \alert{OR~Library} of benchmark datasets from different fields of operations research (OR)%
\item<5-> He also provides several example instances of the JSSP at \url{http://people.brunel.ac.uk/~mastjjb/jeb/orlib/jobshopinfo.html}.%
\item<6-> More information about these instances has been collected by \mbox{van Hoorn\cite{vH2015JSIAS,vH2018TCSOBOBIOTJSSP}} at \url{http://jobshop.jjvh.nl}.
\item<7-> What do such JSSP instances look like?%
\end{itemize}%
\end{frame}%
%
\begin{frame}[b]%
\frametitle{Demo Instance}%
%
\definecolor{c1}{RGB}{165,45,132}%
\definecolor{c2}{RGB}{244,114,22}%
\definecolor{c3}{RGB}{46,49,146}%
\definecolor{c4}{RGB}{99,194,13}%
\definecolor{c5}{RGB}{236,0,140}%
%
\locateGraphic{1}{width=0.98\paperwidth}{graphics/instance/demo_instance_clear}{0.01}{0.19}%
\locateGraphic{2}{width=0.98\paperwidth}{graphics/instance/demo_instance_jobs}{0.01}{0.19}%
\locateGraphic{3}{width=0.98\paperwidth}{graphics/instance/demo_instance_machines}{0.01}{0.19}%
\locateGraphic{4,8}{width=0.98\paperwidth}{graphics/instance/demo_instance_job_1}{0.01}{0.19}%
\locateGraphic{5}{width=0.98\paperwidth}{graphics/instance/demo_instance_job_2}{0.01}{0.19}%
\locateGraphic{6}{width=0.98\paperwidth}{graphics/instance/demo_instance_job_3}{0.01}{0.19}%
\locateGraphic{7}{width=0.98\paperwidth}{graphics/instance/demo_instance_job_4}{0.01}{0.19}%
%
\locateGraphic{9}{width=0.98\paperwidth}{graphics/instance/demo_instance_job_1_1}{0.01}{0.19}%
\locateGraphic{10}{width=0.98\paperwidth}{graphics/instance/demo_instance_job_1_2}{0.01}{0.19}%
\locateGraphic{11}{width=0.98\paperwidth}{graphics/instance/demo_instance_job_1_3}{0.01}{0.19}%
\locateGraphic{12}{width=0.98\paperwidth}{graphics/instance/demo_instance_job_1_4}{0.01}{0.19}%
\locateGraphic{13}{width=0.98\paperwidth}{graphics/instance/demo_instance_job_1_5}{0.01}{0.19}%
%
\locateGraphic{14}{width=0.98\paperwidth}{graphics/instance/demo_instance_job_2_1}{0.01}{0.19}%
\locateGraphic{15}{width=0.98\paperwidth}{graphics/instance/demo_instance_job_2_2}{0.01}{0.19}%
\locateGraphic{16}{width=0.98\paperwidth}{graphics/instance/demo_instance_job_2_3}{0.01}{0.19}%
\locateGraphic{17}{width=0.98\paperwidth}{graphics/instance/demo_instance_job_2_4}{0.01}{0.19}%
\locateGraphic{18}{width=0.98\paperwidth}{graphics/instance/demo_instance_job_2_5}{0.01}{0.19}%
%
\locateGraphic{19}{width=0.98\paperwidth}{graphics/instance/demo_instance_job_3_5}{0.01}{0.19}%
\locateGraphic{20}{width=0.98\paperwidth}{graphics/instance/demo_instance_job_4_5}{0.01}{0.19}%
%
\locateGraphic{21}{width=0.98\paperwidth}{graphics/instance/demo_instance}{0.01}{0.19}%
%
\only<9-13>{%
Job 0 first needs to be processed by \textcolor{c1}{machine 0 for 10 time units}%
\uncover<10->{, it then goes to \textcolor{c2}{machine 1 for 20 time units}%
\uncover<11->{, it then goes to \textcolor{c3}{machine 2 for 20 time units}%
\uncover<12->{, it then goes to \textcolor{c4}{machine 3 for 40 time units}%
\uncover<13->{, and finally it goes to \textcolor{c5}{machine 4 for 10 time units}.%
}}}}}%
%
\only<14-18>{%
Similarly, Job 1 first needs to be processed by \textcolor{c1}{machine 1 for 20 time units}%
\uncover<15->{, it then goes to \textcolor{c2}{machine 0 for 10 time units}%
\uncover<16->{, it then goes to \textcolor{c3}{machine 3 for 30 time units}%
\uncover<17->{, it then goes to \textcolor{c4}{machine 2 for 50 time units}%
\uncover<18->{, and finally it goes to \textcolor{c5}{machine 4 for 30 time units}.%
}}}}}%
\only<19>{%
Job 2 first needs to be processed by \textcolor{c1}{machine 2 for 30 time units}%
, it then goes to \textcolor{c2}{machine 1 for 20 time units}%
, it then goes to \textcolor{c3}{machine 4 for 12 time units}%
, it then goes to \textcolor{c4}{machine 3 for 40 time units}%
, and finally it goes to \textcolor{c5}{machine 0 for 10 time units}.%
}%
\only<20>{%
And Job 3 first needs to be processed by \textcolor{c1}{machine 4 for 50 time units}%
, it then goes to \textcolor{c2}{machine 3 for 30 time units}%
, it then goes to \textcolor{c3}{machine 2 for 15 time units}%
, it then goes to \textcolor{c4}{machine 0 for 20 time units}%
, and finally it goes to \textcolor{c5}{machine 1 for 15 time units}.%
}%
\strut\\\strut\medskip\strut%
%
\end{frame}%
%
\begin{frame}[t]%
\frametitle{Instance \texttt{abz7}}%
Instance \texttt{abz7} by Adams et~al.\cite{ABZ1988TSBPFJSS}%
\locateGraphic{}{width=0.95\paperwidth}{graphics/instance/abz7}{0.02}{0.25}%
\end{frame}%
%
\begin{frame}[t]%
\frametitle{Instance \texttt{la24}}%
Instance \texttt{la24} by Lawrence\cite{L1998RCPSAEIOHSTS}.%
\locateGraphic{}{width=0.9\paperwidth}{graphics/instance/la24}{0.05}{0.275}%
\end{frame}%
%
\begin{frame}[t]%
\frametitle{Instance \texttt{swv15}}%
Instance \texttt{swv15} by Storer et~al.\cite{SWV1992NSSFSPWATJSS}%
\locateGraphic{}{width=0.45\paperwidth}{graphics/instance/swv15}{0.4}{0.2}%
\end{frame}%
%
\begin{frame}[t]%
\frametitle{Instance \texttt{yn4}}%
Instance \texttt{yn4} by Yamada and Nakano\cite{YN1992AGAATLSJSI}.%
\locateGraphic{}{width=0.95\paperwidth}{graphics/instance/yn4}{0.02}{0.25}%
\end{frame}%
%
\begin{frame}[containsverbatim,fragile]%
\frametitle{Problem Instance Data in Java}%
\begin{itemize}%
\item How can we represent such data in Java program code?%
\end{itemize}%
\uncover<2->{%
\listing{0.85}{0.95}{language=myJava,mathescape=false}{code/JSSPInstance.java}%
}%
\end{frame}
%
%
\section{Solution Space}%
%
\begin{frame}[t]%
\frametitle{Output: Candidate Solutions and Solution Space~$\solutionSpace$}%
\only<-4,9->{%
\begin{itemize}%
\item We now know how a problem instance of the JSSP looks like, i.e., the \alert{input} we get.%
\item<2-> But what \alert{output} should we produce?%
\item<3-> In other words, what is a solution for an instance of the JSSP?%
\item<4-> Basically, a Gantt Chart\cite{W2003GCACA,K2000SORCP}.%
\item<9-> A Gantt chart is a diagram which assigns each sub-job on each machine a start and end time.%
\item<10-> The solution space~$\solutionSpace$ is the set of all possible feasible solutions for one JSSP instance.%
\item<11-> One possible solution is called \alert{candidate solution} and it can be illustrated as Gantt chart.%
\end{itemize}%
}%
\only<5-8>{%
\begin{center}
\medskip%
\only<-5>{\large{one possible solution for the \texttt{demo} instance}, illustrated as Gantt chart}%
\only<6>{\large{one possible solution for the \texttt{la24} instance}, illustrated as Gantt chart}%
\only<7>{\large{one possible solution for the \texttt{yn4} instance}, illustrated as Gantt chart}%
\only<8->{\large{one possible solution for the \texttt{swv15} instance}, illustrated as Gantt chart}%
\end{center}%
}%
\locateGraphic{5}{width=0.9\paperwidth}{graphics/gantt/gantt_demo}{0.05}{0.286}%
\locateGraphic{6}{width=0.9\paperwidth}{graphics/gantt/gantt_la24}{0.05}{0.286}%
\locateGraphic{7}{width=0.9\paperwidth}{graphics/gantt/gantt_yn4}{0.05}{0.286}%
\locateGraphic{8}{width=0.9\paperwidth}{graphics/gantt/gantt_swv15}{0.05}{0.286}%
\end{frame}%
%
\begin{frame}[t]%
\frametitle{As Java Class}%
\only<-3>{%
\begin{itemize}%
\item We now need to represent this information as a Java class.%
\item<3-> Each of the~$\jsspMachines$ \jcodeil{int[]} lists in \codeil{schedule} holds~$\jsspJobs$ operations for each machine as three values jobID, start time, end time, i.e., has length~$3\jsspJobs$.%
\end{itemize}%
\uncover<2-3>{%
\listing{0.85}{0.95}{language=myJava,mathescape=false}{code/JSSPCandidateSolution.java}%
}%
}%
%
\locateGraphic{4}{width=0.8\paperwidth}{graphics/candidate_solution/demo_candidate_solution}{0.075}{0.11}%
\locateGraphic{5}{width=0.8\paperwidth}{graphics/candidate_solution/demo_candidate_solution_machines_colored}{0.11}{0.075}%
\locateGraphic{6}{width=0.8\paperwidth}{graphics/candidate_solution/demo_candidate_solution_0_01}{0.075}{0.125}%
\locateGraphic{7}{width=0.8\paperwidth}{graphics/candidate_solution/demo_candidate_solution_0_02}{0.075}{0.125}%
\locateGraphic{8}{width=0.8\paperwidth}{graphics/candidate_solution/demo_candidate_solution_0_03}{0.075}{0.125}%
\locateGraphic{9}{width=0.8\paperwidth}{graphics/candidate_solution/demo_candidate_solution_0_04}{0.075}{0.125}%
\locateGraphic{10}{width=0.8\paperwidth}{graphics/candidate_solution/demo_candidate_solution_0_05}{0.075}{0.125}%
\locateGraphic{11}{width=0.8\paperwidth}{graphics/candidate_solution/demo_candidate_solution_0_06}{0.075}{0.125}%
\locateGraphic{12}{width=0.8\paperwidth}{graphics/candidate_solution/demo_candidate_solution_0_07}{0.075}{0.125}%
\locateGraphic{13}{width=0.8\paperwidth}{graphics/candidate_solution/demo_candidate_solution_0_08}{0.075}{0.125}%
\locateGraphic{14}{width=0.8\paperwidth}{graphics/candidate_solution/demo_candidate_solution_0_09}{0.075}{0.125}%
\locateGraphic{15}{width=0.8\paperwidth}{graphics/candidate_solution/demo_candidate_solution_0_10}{0.075}{0.125}%
\locateGraphic{16}{width=0.8\paperwidth}{graphics/candidate_solution/demo_candidate_solution_0_11}{0.075}{0.125}%
\locateGraphic{17}{width=0.8\paperwidth}{graphics/candidate_solution/demo_candidate_solution_0_12}{0.075}{0.125}%
\locateGraphic{18}{width=0.8\paperwidth}{graphics/candidate_solution/demo_candidate_solution_1_01}{0.075}{0.125}%
\locateGraphic{19}{width=0.8\paperwidth}{graphics/candidate_solution/demo_candidate_solution_1_02}{0.075}{0.125}%
\locateGraphic{20}{width=0.8\paperwidth}{graphics/candidate_solution/demo_candidate_solution_1_03}{0.075}{0.125}%
\locateGraphic{21}{width=0.8\paperwidth}{graphics/candidate_solution/demo_candidate_solution_1_04}{0.075}{0.125}%
\locateGraphic{22}{width=0.8\paperwidth}{graphics/candidate_solution/demo_candidate_solution_1_05}{0.075}{0.125}%
\locateGraphic{23}{width=0.8\paperwidth}{graphics/candidate_solution/demo_candidate_solution_1_06}{0.075}{0.125}%
\locateGraphic{24}{width=0.8\paperwidth}{graphics/candidate_solution/demo_candidate_solution_1_07}{0.075}{0.125}%
\locateGraphic{25}{width=0.8\paperwidth}{graphics/candidate_solution/demo_candidate_solution_1_08}{0.075}{0.125}%
\locateGraphic{26}{width=0.8\paperwidth}{graphics/candidate_solution/demo_candidate_solution_1_09}{0.075}{0.125}%
\locateGraphic{27}{width=0.8\paperwidth}{graphics/candidate_solution/demo_candidate_solution_1_10}{0.075}{0.125}%
\locateGraphic{28}{width=0.8\paperwidth}{graphics/candidate_solution/demo_candidate_solution_1_11}{0.075}{0.125}%
\locateGraphic{29}{width=0.8\paperwidth}{graphics/candidate_solution/demo_candidate_solution_1_12}{0.075}{0.125}%
\end{frame}%
%
\section{Objective Function}%
%
\begin{frame}[t]%
\frametitle{Solution Quality}%
\begin{itemize}%
\only<-6,17-18>{%
\item So we have identified what the possible solutions to our problems are and know how to store them in a data structure.%
\item<2-> How do we rate the quality of a solution?%
\item<3-> A Gantt chart $\solspel_{1}\in\solutionSpace$ is a better solution to our problem than another chart $\solspel_{2}\in\solutionSpace$ if \textcolor<18>{red}{it allows us to complete our work faster}.%
}%
\only<-18>{%
\item<4-> The \alert<4>{objective function} $\objf:\solutionSpace\mapsto\realNumbers$ is the \alert{makespan}\uncover<5->{, the time when the last sub-job is completed\uncover<6->{, the right-most edge of any bar in the Gantt chart.}}%
\item<17-> This objective function is subject to minimization: smaller values are better.%
}%
\item<18-> \alert{A Gantt chart $\solspel_{1}\in\solutionSpace$ is a better solution to our problem than another chart $\solspel_{2}\in\solutionSpace$ if \textcolor<18>{red}{$\objf(\solspel_{1})<\objf(\solspel_{2})$}.}%
\end{itemize}%
%
\locateGraphic{7}{width=0.9\paperwidth}{graphics/makespan/no_makespan_demo}{0.05}{0.3}%
\locateGraphic{8}{width=0.9\paperwidth}{graphics/makespan/makespan_demo}{0.05}{0.3}%
\locateGraphic{9}{width=0.9\paperwidth}{graphics/makespan/no_makespan_abz7}{0.05}{0.3}%
\locateGraphic{10}{width=0.9\paperwidth}{graphics/makespan/makespan_abz7}{0.05}{0.3}%
\locateGraphic{11}{width=0.9\paperwidth}{graphics/makespan/no_makespan_la24}{0.05}{0.3}%
\locateGraphic{12}{width=0.9\paperwidth}{graphics/makespan/makespan_la24}{0.05}{0.3}%
\locateGraphic{13}{width=0.9\paperwidth}{graphics/makespan/no_makespan_swv15}{0.05}{0.3}%
\locateGraphic{14}{width=0.9\paperwidth}{graphics/makespan/makespan_swv15}{0.05}{0.3}%
\locateGraphic{15}{width=0.9\paperwidth}{graphics/makespan/no_makespan_yn4}{0.05}{0.3}%
\locateGraphic{16}{width=0.9\paperwidth}{graphics/makespan/makespan_yn4}{0.05}{0.3}%
%
\locateGraphic{19}{width=0.5\paperwidth}{graphics/makespan/makespan_good_abz7}{0.15}{0.25}%
\locateGraphic{19}{width=0.5\paperwidth}{graphics/makespan/makespan_bad_abz7}{0.35}{0.6}%
\locateGraphic{20}{width=0.9\paperwidth}{graphics/makespan/makespan_good_abz7}{0.05}{0.3}%
%
\locateGraphic{21}{width=0.5\paperwidth}{graphics/makespan/makespan_bad_la24}{0.15}{0.25}%
\locateGraphic{21}{width=0.5\paperwidth}{graphics/makespan/makespan_good_la24}{0.35}{0.6}%
\locateGraphic{22}{width=0.9\paperwidth}{graphics/makespan/makespan_good_yn4}{0.05}{0.3}%
%
\locateGraphic{23}{width=0.5\paperwidth}{graphics/makespan/makespan_bad_swv15}{0.15}{0.25}%
\locateGraphic{23}{width=0.5\paperwidth}{graphics/makespan/makespan_good_swv15}{0.35}{0.6}%
\locateGraphic{24}{width=0.9\paperwidth}{graphics/makespan/makespan_good_swv15}{0.05}{0.3}%
%
\locateGraphic{25}{width=0.5\paperwidth}{graphics/makespan/makespan_good_yn4}{0.15}{0.25}%
\locateGraphic{25}{width=0.5\paperwidth}{graphics/makespan/makespan_bad_yn4}{0.35}{0.6}%
\locateGraphic{26}{width=0.9\paperwidth}{graphics/makespan/makespan_good_yn4}{0.05}{0.3}%
%
\end{frame}%
%
\begin{frame}[containsverbatim]%
\frametitle{An Interface for Objective Functions in Java}%
\centering%
\listing{0.85}{0.95}{language=myJava,mathescape=false}{code/JSSPInstance.java}%
\end{frame}%
%
\begin{frame}[containsverbatim]%
\frametitle{The JSSP Objective Function in Java}%
\centering%
\listing{0.85}{0.95}{language=myJava,mathescape=false}{code/JSSPMakespanObjectiveFunction.java}%
\end{frame}%
%
\begin{frame}%
\frametitle{The Global Optimum $\globalOptimum{\solspel}$ in $\solutionSpace$}%
\only<-4,6->{%
\begin{itemize}%
\item There must be at least one \alert{globally optimal} solution $\globalOptimum{\solspel}$\only<-1>{.}%
\uncover<2->{ for which $\objf(\globalOptimum{\solspel})\leq\objf(\solspel)\;\forall\solspel\in\solutionSpace$ holds.}%
\item<3-> How do we find such a solution?%
\item<4-> We know the problem is \mbox{$\NPprefix$-hard\cite{K1972RACP,C1971TCOTPP}}, so any algorithm that \alert{guarantees} that it will always find this solution \alert{may sometimes} need a runtime exponential in~$\jsspMachines$ or~$\jsspJobs$ in the worst case.%
\item<6-> So we cannot \alert{guarantee} to always find the best possible solution for a normal-sized JSSP in \alert{reasonable time}.%
\item<7-> What we can always do is search in~$\solutionSpace$ and hope to get as close to $\globalOptimum{\solspel}$ within reasonable time as possible.%
\item<8-> If we can find a solution with a slightly larger makespan than the best possible solution, but we can get it within a few minutes, that would also be nice\dots%
\end{itemize}%
}%
\locateGraphic{5}{width=0.98\paperwidth}{../01_introduction/graphics/function_growth/function_growth}{0.01}{0.2}%
\end{frame}%
%

\section{From Solution Space to Search Space}%
%
\begin{frame}%
\frametitle{Feasibility of Solutions}%
\only<-4,11,15,19,23,27>{%
\begin{itemize}%
\item So what do we need to consider when searching in~$\solutionSpace$?%
\item<2-> A candidate solution~$\solspel\in\solutionSpace$ is \alert{feasible}, i.e., can actually be ``used,'' if and only if it fulfills all \emph{constraints}.%
\item<3-> Indeed, there are several constraints we need to impose on our Gantt charts\uncover<4->{:%
\begin{enumerate}%
\item all operations of all jobs must be assigned to their respective machines and properly be completed\uncover<11->{,}%
\item<11-> only the jobs and machines specified by the problem instance must occur in the chart\uncover<15->{,}%
\item<15-> an operations must be assigned a time window on its corresponding machine which is exactly as long as the operation needs on that machine\uncover<19->{,}%
\item<19-> the operations cannot intersect or overlap, each machine can only carry out one job at a time\uncover<23->{, and}%
\item<23-> the precedence constraints of the operations must be honored.%
\end{enumerate}%
}%
\item<27-> Only a Gantt chart obeying all of these constraints is feasible, i.e., can be implemented in practice.%
\end{itemize}%
}%
%
%
\locateGraphic{5,8,16,20,24}{width=0.7\paperwidth}{graphics/errors/error}{0.15}{0.2}%
\locateGraphic{6}{width=0.7\paperwidth}{graphics/errors/error_incomplete_1}{0.15}{0.2}%
\locateGraphic{7}{width=0.7\paperwidth}{graphics/errors/error_incomplete_2}{0.15}{0.2}%
\locateGraphic{9}{width=0.7\paperwidth}{graphics/errors/error_machine_1}{0.15}{0.2}%
\locateGraphic{10}{width=0.7\paperwidth}{graphics/errors/error_machine_2}{0.15}{0.2}%
\locateGraphic{12}{width=0.7\paperwidth}{graphics/errors/error_add_machine_0}{0.15}{0.2}%
\locateGraphic{13}{width=0.7\paperwidth}{graphics/errors/error_add_machine_1}{0.15}{0.2}%
\locateGraphic{14}{width=0.7\paperwidth}{graphics/errors/error_add_machine_2}{0.15}{0.2}%
\locateGraphic{17}{width=0.7\paperwidth}{graphics/errors/error_time_1}{0.15}{0.2}%
\locateGraphic{18}{width=0.7\paperwidth}{graphics/errors/error_time_2}{0.15}{0.2}%
\locateGraphic{21}{width=0.7\paperwidth}{graphics/errors/error_overlap_1}{0.15}{0.2}%
\locateGraphic{22}{width=0.7\paperwidth}{graphics/errors/error_overlap_2}{0.15}{0.2}%
\locateGraphic{25}{width=0.7\paperwidth}{graphics/errors/error_precedence_1}{0.15}{0.2}%
\locateGraphic{26}{width=0.7\paperwidth}{graphics/errors/error_precedence_2}{0.15}{0.2}%
%
\end{frame}%
%
%
\begin{frame}%
\frametitle{Hardships when Searching in~$\solutionSpace$}%
\only<-3,82->{%
\begin{itemize}%
\item So how do we search in the space of Gantt charts?%
\item<2-> We need to create Gantt charts that fulfill all the constraints.%
\item<3-> For different \alert{instances}, different solutions are \alert{feasible}!%
\item<82-> Writing Java code that works directly on the Gantt charts is cumbersome and error-prone.%
\item<83-> Actually, the vast majority of possible Gantt charts will often be infeasible and have deadlocks\dots%
\item<84-> We would like to have a handy \alert{representation} for Gantt charts.%
\item<85-> The representation should allow us to easy create and modify the candidate solutions.%
\item<86-> \alert{Solution}: We develop a data structure~$\searchSpace$ which we can handle easily and which can \alert{always} be translated to feasible Gantt charts by a mapping $\repMap:\searchSpace\mapsto\solutionSpace$.%
\end{itemize}%
}%
%
\locateGraphic{4}{width=0.98\paperwidth}{graphics/feasible/jssp_gantt_feasible_01}{0.01}{0.2}%
\locateGraphic{5}{width=0.98\paperwidth}{graphics/feasible/jssp_gantt_feasible_02}{0.01}{0.2}%
\locateGraphic{6}{width=0.98\paperwidth}{graphics/feasible/jssp_gantt_feasible_03}{0.01}{0.2}%
\locateGraphic{7}{width=0.98\paperwidth}{graphics/feasible/jssp_gantt_feasible_04}{0.01}{0.2}%
\locateGraphic{8}{width=0.98\paperwidth}{graphics/feasible/jssp_gantt_feasible_05}{0.01}{0.2}%
\locateGraphic{9}{width=0.98\paperwidth}{graphics/feasible/jssp_gantt_feasible_06}{0.01}{0.2}%
\locateGraphic{10}{width=0.98\paperwidth}{graphics/feasible/jssp_gantt_feasible_07}{0.01}{0.2}%
\locateGraphic{11}{width=0.98\paperwidth}{graphics/feasible/jssp_gantt_feasible_08}{0.01}{0.2}%
\locateGraphic{12}{width=0.98\paperwidth}{graphics/feasible/jssp_gantt_feasible_09}{0.01}{0.2}%
\locateGraphic{13}{width=0.98\paperwidth}{graphics/feasible/jssp_gantt_feasible_10}{0.01}{0.2}%
\locateGraphic{14}{width=0.98\paperwidth}{graphics/feasible/jssp_gantt_feasible_11}{0.01}{0.2}%
\locateGraphic{15}{width=0.98\paperwidth}{graphics/feasible/jssp_gantt_feasible_12}{0.01}{0.2}%
\locateGraphic{16}{width=0.98\paperwidth}{graphics/feasible/jssp_gantt_feasible_13}{0.01}{0.2}%
\locateGraphic{17}{width=0.98\paperwidth}{graphics/feasible/jssp_gantt_feasible_14}{0.01}{0.2}%
\locateGraphic{18}{width=0.98\paperwidth}{graphics/feasible/jssp_gantt_feasible_15}{0.01}{0.2}%
\locateGraphic{19}{width=0.98\paperwidth}{graphics/feasible/jssp_gantt_feasible_16}{0.01}{0.2}%
\locateGraphic{20}{width=0.98\paperwidth}{graphics/feasible/jssp_gantt_feasible_17}{0.01}{0.2}%
\locateGraphic{21}{width=0.98\paperwidth}{graphics/feasible/jssp_gantt_feasible_18}{0.01}{0.2}%
\locateGraphic{22}{width=0.98\paperwidth}{graphics/feasible/jssp_gantt_feasible_19}{0.01}{0.2}%
\locateGraphic{23}{width=0.98\paperwidth}{graphics/feasible/jssp_gantt_feasible_20}{0.01}{0.2}%
\locateGraphic{24}{width=0.98\paperwidth}{graphics/feasible/jssp_gantt_feasible_21}{0.01}{0.2}%
\locateGraphic{25}{width=0.98\paperwidth}{graphics/feasible/jssp_gantt_feasible_22}{0.01}{0.2}%
\locateGraphic{26}{width=0.98\paperwidth}{graphics/feasible/jssp_gantt_feasible_23}{0.01}{0.2}%
\locateGraphic{27}{width=0.98\paperwidth}{graphics/feasible/jssp_gantt_feasible_24}{0.01}{0.2}%
\locateGraphic{28}{width=0.98\paperwidth}{graphics/feasible/jssp_gantt_feasible_25}{0.01}{0.2}%
\locateGraphic{29}{width=0.98\paperwidth}{graphics/feasible/jssp_gantt_feasible_26}{0.01}{0.2}%
\locateGraphic{30}{width=0.98\paperwidth}{graphics/feasible/jssp_gantt_feasible_27}{0.01}{0.2}%
\locateGraphic{31}{width=0.98\paperwidth}{graphics/feasible/jssp_gantt_feasible_28}{0.01}{0.2}%
\locateGraphic{32}{width=0.98\paperwidth}{graphics/feasible/jssp_gantt_feasible_29}{0.01}{0.2}%
\locateGraphic{33}{width=0.98\paperwidth}{graphics/feasible/jssp_gantt_feasible_30}{0.01}{0.2}%
\locateGraphic{34}{width=0.98\paperwidth}{graphics/feasible/jssp_gantt_feasible_31}{0.01}{0.2}%
\locateGraphic{35}{width=0.98\paperwidth}{graphics/feasible/jssp_gantt_feasible_32}{0.01}{0.2}%
\locateGraphic{36}{width=0.98\paperwidth}{graphics/feasible/jssp_gantt_feasible_33}{0.01}{0.2}%
\locateGraphic{37}{width=0.98\paperwidth}{graphics/feasible/jssp_gantt_feasible}{0.01}{0.2}%
%
%
\locateGraphic{38}{width=0.98\paperwidth}{graphics/feasible/jssp_gantt_infeasible_01}{0.01}{0.2}%
\locateGraphic{39}{width=0.98\paperwidth}{graphics/feasible/jssp_gantt_infeasible_02}{0.01}{0.2}%
\locateGraphic{40}{width=0.98\paperwidth}{graphics/feasible/jssp_gantt_infeasible_03}{0.01}{0.2}%
\locateGraphic{41}{width=0.98\paperwidth}{graphics/feasible/jssp_gantt_infeasible_04}{0.01}{0.2}%
\locateGraphic{42}{width=0.98\paperwidth}{graphics/feasible/jssp_gantt_infeasible_05}{0.01}{0.2}%
\locateGraphic{43}{width=0.98\paperwidth}{graphics/feasible/jssp_gantt_infeasible_06}{0.01}{0.2}%
\locateGraphic{44}{width=0.98\paperwidth}{graphics/feasible/jssp_gantt_infeasible_07}{0.01}{0.2}%
\locateGraphic{45}{width=0.98\paperwidth}{graphics/feasible/jssp_gantt_infeasible_08}{0.01}{0.2}%
\locateGraphic{46}{width=0.98\paperwidth}{graphics/feasible/jssp_gantt_infeasible_09}{0.01}{0.2}%
\locateGraphic{47}{width=0.98\paperwidth}{graphics/feasible/jssp_gantt_infeasible_10}{0.01}{0.2}%
\locateGraphic{48}{width=0.98\paperwidth}{graphics/feasible/jssp_gantt_infeasible_11}{0.01}{0.2}%
\locateGraphic{49}{width=0.98\paperwidth}{graphics/feasible/jssp_gantt_infeasible_12}{0.01}{0.2}%
\locateGraphic{50}{width=0.98\paperwidth}{graphics/feasible/jssp_gantt_infeasible_13}{0.01}{0.2}%
\locateGraphic{51}{width=0.98\paperwidth}{graphics/feasible/jssp_gantt_infeasible_14}{0.01}{0.2}%
\locateGraphic{52}{width=0.98\paperwidth}{graphics/feasible/jssp_gantt_infeasible_15}{0.01}{0.2}%
\locateGraphic{53}{width=0.98\paperwidth}{graphics/feasible/jssp_gantt_infeasible_16}{0.01}{0.2}%
\locateGraphic{54}{width=0.98\paperwidth}{graphics/feasible/jssp_gantt_infeasible_17}{0.01}{0.2}%
\locateGraphic{55}{width=0.98\paperwidth}{graphics/feasible/jssp_gantt_infeasible_18}{0.01}{0.2}%
\locateGraphic{56}{width=0.98\paperwidth}{graphics/feasible/jssp_gantt_infeasible_19}{0.01}{0.2}%
\locateGraphic{57}{width=0.98\paperwidth}{graphics/feasible/jssp_gantt_infeasible_20}{0.01}{0.2}%
\locateGraphic{58}{width=0.98\paperwidth}{graphics/feasible/jssp_gantt_infeasible_21}{0.01}{0.2}%
\locateGraphic{59}{width=0.98\paperwidth}{graphics/feasible/jssp_gantt_infeasible_22}{0.01}{0.2}%
\locateGraphic{60}{width=0.98\paperwidth}{graphics/feasible/jssp_gantt_infeasible_23}{0.01}{0.2}%
\locateGraphic{61}{width=0.98\paperwidth}{graphics/feasible/jssp_gantt_infeasible_24}{0.01}{0.2}%
\locateGraphic{62}{width=0.98\paperwidth}{graphics/feasible/jssp_gantt_infeasible_25}{0.01}{0.2}%
\locateGraphic{63}{width=0.98\paperwidth}{graphics/feasible/jssp_gantt_infeasible_26}{0.01}{0.2}%
\locateGraphic{64}{width=0.98\paperwidth}{graphics/feasible/jssp_gantt_infeasible_27}{0.01}{0.2}%
\locateGraphic{65}{width=0.98\paperwidth}{graphics/feasible/jssp_gantt_infeasible_28}{0.01}{0.2}%
\locateGraphic{66-75}{width=0.98\paperwidth}{graphics/feasible/jssp_gantt_infeasible_29}{0.01}{0.2}%
%
\locate{67-75}{\parbox{0.475\paperwidth}{%%
\only<-69>{\textcolor<68->{gray}{Machine~0 should begin by doing job~1. }}%
\only<68-70>{\textcolor<69->{gray}{Job~1 can only start on machine~0 after it has been finished on machine~1. }}%
\only<69-70>{\textcolor<70->{gray}{At machine~1, we should begin with job~0. }}%
\only<70>{Before job~0 can be put on machine~1, it must go through machine~0. }%
\only<71>{So job~1 cannot go to machine~0 until it has passed through machine~1, but in order to be executed on machine~1, job~0 needs to be finished there first. }%
\only<72>{Job~0 cannot begin on machine~1 until it has been passed through machine~0, but it cannot be executed there, because job~1 needs to be finished there first. }%
\only<73-74>{\textcolor<74->{gray}{A cyclic blockage has appeared: no job can be executed on any machine if we follow this schedule. }}%
\only<74-75>{\textcolor<75->{gray}{This is called a \textcolor<19>{red}{deadlock}. }}%
\only<75>{The schedule is infeasible, because it cannot be executed or written down without breaking the precedence constraint. }%
}}{0.5}{0.2}%
%
\locateGraphic{76}{width=0.98\paperwidth}{graphics/feasible/jssp_gantt_infeasible_30}{0.01}{0.2}%
\locateGraphic{77}{width=0.98\paperwidth}{graphics/feasible/jssp_gantt_infeasible_31}{0.01}{0.2}%
\locateGraphic{78}{width=0.98\paperwidth}{graphics/feasible/jssp_gantt_infeasible_32}{0.01}{0.2}%
\locateGraphic{79}{width=0.98\paperwidth}{graphics/feasible/jssp_gantt_infeasible_33}{0.01}{0.2}%
\locateGraphic{80}{width=0.98\paperwidth}{graphics/feasible/jssp_gantt_infeasible_34}{0.01}{0.2}%
\locateGraphic{81}{width=0.98\paperwidth}{graphics/feasible/jssp_gantt_infeasible}{0.01}{0.2}%
%
\end{frame}%
%
\begin{frame}[t]%
\frametitle{The Search Space~$\searchSpace$}%
\begin{itemize}%
\item The solution space~$\solutionSpace$ is complicated and constrained.%
\item<2-> In a real-world JSSP, there would even be more issues, such as job- and machine-specific setup times and transfer times\dots%
\item<3-> If we would have a valid Gantt chart~$\solspel\in\solutionSpace$, then trying to improve it would be quite complicated.%
\item<4-> If we imagine the space~$\solutionSpace$ of possible Gantt charts for a JSSP, then searching through this space in some kind of targeted way would be complicated.%
\item<5-> We want to search in a simpler space that we can easily understand\only<-5>{.}\uncover<6->{, where we do not need to worry about the constraints and feasibility.}%
\item<6-> This space is therefore called the search space~$\searchSpace$.%
\item<7-> Of course, $\searchSpace$ must somehow be related to $\solutionSpace$\only<-7>{.}\uncover<8->{: We need a representation mapping~$\repMap:\searchSpace\mapsto\solutionSpace$ which translates from~$\searchSpace$ to~$\solutionSpace$.}%
\end{itemize}%
\end{frame}%
%
\begin{frame}[t]%
\frametitle{One Search Space~$\searchSpace$ for the JSSP}%
\only<-2,5->{%
\begin{itemize}%
\item So how could a simple search space~$\searchSpace$ for the JSSP look like?%
\item<2-> Let us revisit the demo problem instance.%
%
\item<5-> Ideally, we want to \alert{encode} this two-dimensional structure in \only<-5>{something very simple}\only<6->{a simple one-dimensional string of integer numbers}.%
\item<7-> In the demo, we have $\jsspMachines=5$ machines and $\jsspJobs=4$ jobs.%
\item<8-> We could give each of the $\jsspMachines*\jsspJobs=20$ operation one ID, a number in $0\dots 19$.%
\only<-11>{%
\item<9-> Then, a linear string containing a permutation of these IDs could denote the exact processing order of the operations.
\item<10-> We could easily translate such strings to Gantt charts\only<-10>{.}\uncover<11->{, but we could end up with infeasible solutions and deadlocks or a string telling us to do the second operation of a job before the first one\dots}%
}%
\item<12-> How can we use a linear encoding without deadlocks?%
\item<13-> Each job has $\jsspMachines=5$ operations that must be distributed to the machines in the sequence prescribed in the problem instance data.%
\item<14-> We \alert{know} the order of the operations per job\only<-14>{.}\uncover<15->{ $\Longrightarrow$ we do not need to encode it.}%
\item<16-> We just include each job id $\jsspMachines$~times in the string.\cite{GTK1994SJSSPBGA,B1995AGPATJSSWGA,BMK1996OPRFSP,SIS1997NESFSJSPBGA}%
\item<17-> The first occurrence of a job's ID stands for its first operation, the second occurrence for the second operation, and so on.%
\item<18-> This way, we will always have the operations in the right order.%
\end{itemize}%
}%
%
\locateGraphic{3}{width=0.95\paperwidth}{graphics/instance/demo_instance}{0.025}{0.27}%
\locate{3}{\parbox{0.7\paperwidth}{This is information that we have, which does not need to be stored in the elements~$\sespel\in\searchSpace$.}}{0.15}{0.8}%
\locateGraphic{4}{width=0.9\paperwidth}{graphics/gantt/gantt_demo}{0.05}{0.16}%
\locate{4}{\parbox{0.7\paperwidth}{The instance data~$\instance$ and the data from one point~$\sespel\in\searchSpace$ should, together, encode such a Gantt chart~$\solspel\in\solutionSpace$.}}{0.15}{0.8}%
%
\end{frame}%
%
\begin{frame}[t]%
\frametitle{Demo Example for the Search Space}%
\locateGraphic{1}{width=0.85\paperwidth}{graphics/representation/representation_demo_instance_01}{0.075}{0.12}%
\locateGraphic{2}{width=0.85\paperwidth}{graphics/representation/representation_demo_instance_02}{0.075}{0.12}%
\locateGraphic{3}{width=0.85\paperwidth}{graphics/representation/representation_demo_instance_03}{0.075}{0.12}%
\locateGraphic{4}{width=0.85\paperwidth}{graphics/representation/representation_demo_instance_04}{0.075}{0.12}%
\locateGraphic{5}{width=0.85\paperwidth}{graphics/representation/representation_demo_instance_05}{0.075}{0.12}%
\locateGraphic{6}{width=0.85\paperwidth}{graphics/representation/representation_demo_instance_06}{0.075}{0.12}%
\locateGraphic{7}{width=0.85\paperwidth}{graphics/representation/representation_demo_instance_07}{0.075}{0.12}%
\locateGraphic{8}{width=0.85\paperwidth}{graphics/representation/representation_demo_instance_08}{0.075}{0.12}%
\locateGraphic{9}{width=0.85\paperwidth}{graphics/representation/representation_demo_instance_09}{0.075}{0.12}%
\locateGraphic{10}{width=0.85\paperwidth}{graphics/representation/representation_demo_instance_10}{0.075}{0.12}%
\locateGraphic{11}{width=0.85\paperwidth}{graphics/representation/representation_demo_instance_11}{0.075}{0.12}%
\locateGraphic{12}{width=0.85\paperwidth}{graphics/representation/representation_demo_instance_12}{0.075}{0.12}%
\locateGraphic{13}{width=0.85\paperwidth}{graphics/representation/representation_demo_instance_13}{0.075}{0.12}%
\locateGraphic{14}{width=0.85\paperwidth}{graphics/representation/representation_demo_instance_14}{0.075}{0.12}%
\locateGraphic{15}{width=0.85\paperwidth}{graphics/representation/representation_demo_instance_15}{0.075}{0.12}%
\locateGraphic{16}{width=0.85\paperwidth}{graphics/representation/representation_demo_instance_16}{0.075}{0.12}%
\locateGraphic{17}{width=0.85\paperwidth}{graphics/representation/representation_demo_instance_17}{0.075}{0.12}%
\locateGraphic{18}{width=0.85\paperwidth}{graphics/representation/representation_demo_instance_18}{0.075}{0.12}%
\locateGraphic{19}{width=0.85\paperwidth}{graphics/representation/representation_demo_instance_19}{0.075}{0.12}%
\locateGraphic{20}{width=0.85\paperwidth}{graphics/representation/representation_demo_instance_20}{0.075}{0.12}%
\locateGraphic{21}{width=0.85\paperwidth}{graphics/representation/representation_demo_instance_21}{0.075}{0.12}%
\locateGraphic{22}{width=0.85\paperwidth}{graphics/representation/representation_demo_instance_22}{0.075}{0.12}%
\locateGraphic{23}{width=0.85\paperwidth}{graphics/representation/representation_demo_instance_23}{0.075}{0.12}%
\locateGraphic{24}{width=0.85\paperwidth}{graphics/representation/representation_demo_instance_24}{0.075}{0.12}%
\locateGraphic{25}{width=0.85\paperwidth}{graphics/representation/representation_demo_instance_25}{0.075}{0.12}%
\locateGraphic{26}{width=0.85\paperwidth}{graphics/representation/representation_demo_instance_26}{0.075}{0.12}%
\locateGraphic{27}{width=0.85\paperwidth}{graphics/representation/representation_demo_instance_27}{0.075}{0.12}%
\locateGraphic{28}{width=0.85\paperwidth}{graphics/representation/representation_demo_instance_28}{0.075}{0.12}%
\locateGraphic{29}{width=0.85\paperwidth}{graphics/representation/representation_demo_instance_29}{0.075}{0.12}%
\locateGraphic{30}{width=0.85\paperwidth}{graphics/representation/representation_demo_instance_30}{0.075}{0.12}%
\locateGraphic{31}{width=0.85\paperwidth}{graphics/representation/representation_demo_instance_31}{0.075}{0.12}%
\locateGraphic{32}{width=0.85\paperwidth}{graphics/representation/representation_demo_instance_32}{0.075}{0.12}%
\locateGraphic{33}{width=0.85\paperwidth}{graphics/representation/representation_demo_instance_33}{0.075}{0.12}%
\locateGraphic{34}{width=0.85\paperwidth}{graphics/representation/representation_demo_instance_34}{0.075}{0.12}%
\locateGraphic{35}{width=0.85\paperwidth}{graphics/representation/representation_demo_instance_35}{0.075}{0.12}%
\locateGraphic{36}{width=0.85\paperwidth}{graphics/representation/representation_demo_instance_36}{0.075}{0.12}%
\locateGraphic{37}{width=0.85\paperwidth}{graphics/representation/representation_demo_instance_37}{0.075}{0.12}%
\locateGraphic{38}{width=0.85\paperwidth}{graphics/representation/representation_demo_instance_38}{0.075}{0.12}%
\locateGraphic{39}{width=0.85\paperwidth}{graphics/representation/representation_demo_instance_39}{0.075}{0.12}%
\locateGraphic{40}{width=0.85\paperwidth}{graphics/representation/representation_demo_instance_40}{0.075}{0.12}%
\locateGraphic{41}{width=0.85\paperwidth}{graphics/representation/representation_demo_instance_41}{0.075}{0.12}%
\locateGraphic{42}{width=0.85\paperwidth}{graphics/representation/representation_demo_instance_42}{0.075}{0.12}%
\locateGraphic{43}{width=0.85\paperwidth}{graphics/representation/representation_demo_instance_43}{0.075}{0.12}%
\locateGraphic{44}{width=0.85\paperwidth}{graphics/representation/representation_demo_instance_44}{0.075}{0.12}%
\locateGraphic{45}{width=0.85\paperwidth}{graphics/representation/representation_demo_instance_45}{0.075}{0.12}%
\locateGraphic{46}{width=0.85\paperwidth}{graphics/representation/representation_demo_instance_46}{0.075}{0.12}%
\locateGraphic{47}{width=0.85\paperwidth}{graphics/representation/representation_demo_instance_47}{0.075}{0.12}%
\locateGraphic{48}{width=0.85\paperwidth}{graphics/representation/representation_demo_instance_48}{0.075}{0.12}%
\locateGraphic{49}{width=0.85\paperwidth}{graphics/representation/representation_demo_instance_49}{0.075}{0.12}%
\locateGraphic{50}{width=0.85\paperwidth}{graphics/representation/representation_demo_instance_50}{0.075}{0.12}%
\locateGraphic{51}{width=0.85\paperwidth}{graphics/representation/representation_demo_instance_51}{0.075}{0.12}%
\locateGraphic{52}{width=0.85\paperwidth}{graphics/representation/representation_demo_instance_52}{0.075}{0.12}%
\locateGraphic{53}{width=0.85\paperwidth}{graphics/representation/representation_demo_instance_53}{0.075}{0.12}%
\locateGraphic{54}{width=0.85\paperwidth}{graphics/representation/representation_demo_instance_54}{0.075}{0.12}%
\locateGraphic{55}{width=0.85\paperwidth}{graphics/representation/representation_demo_instance_55}{0.075}{0.12}%
\locateGraphic{56}{width=0.85\paperwidth}{graphics/representation/representation_demo_instance_56}{0.075}{0.12}%
\locateGraphic{57}{width=0.85\paperwidth}{graphics/representation/representation_demo_instance_57}{0.075}{0.12}%
\locateGraphic{58}{width=0.85\paperwidth}{graphics/representation/representation_demo_instance_58}{0.075}{0.12}%
\locateGraphic{59}{width=0.85\paperwidth}{graphics/representation/representation_demo_instance_59}{0.075}{0.12}%
\locateGraphic{60}{width=0.85\paperwidth}{graphics/representation/representation_demo_instance_60}{0.075}{0.12}%
\locateGraphic{61}{width=0.85\paperwidth}{graphics/representation/representation_demo_instance_61}{0.075}{0.12}%
\locateGraphic{62}{width=0.85\paperwidth}{graphics/representation/representation_demo_instance_62}{0.075}{0.12}%
\locateGraphic{63}{width=0.85\paperwidth}{graphics/representation/representation_demo_instance_63}{0.075}{0.12}%
\locateGraphic{64}{width=0.85\paperwidth}{graphics/representation/representation_demo_instance_64}{0.075}{0.12}%
\end{frame}%
%

\begin{frame}%
\frametitle{The Search Space~$\searchSpace$}%
\begin{itemize}%
\item We now have search space~$\searchSpace$ with which we can easily represent all reasonable Gantt charts.%
\item<2-> As long as our integer strings of length $\jsspMachines*\jsspJobs$ contain each value in $1\dots\jsspJobs$ exactly~$\jsspMachines$ times, we will always get \alert{feasible} Gantt charts by applying our mapping $\repMap:\searchSpace\mapsto\solutionSpace$!%
\item<3-> We call this the \alert{representation}.%
\item<4-> If necessary, we could also easily add more constraints, such as job-order specific machine setup times, or job/machine specific transport times -- they would all go into the mapping~$\repMap$.%
\end{itemize}%
\end{frame}%
%
\begin{frame}[containsverbatim]%
\frametitle{An Interface for Representation Mappings in Java}%
\listing{0.92}{0.76}{language=myJava,mathescape=false}{code/IRepresentationMapping.java}%
\end{frame}%
%
\begin{frame}[containsverbatim]%
\frametitle{The JSSP Representation Mapping in Java}%
\listing{0.92}{1.33}{language=myJava,mathescape=false}{code/JSSPRepresentationMapping.java}%
\end{frame}%
%
\section{Number of Possible Solutions}%
%
\begin{frame}%
\frametitle{Number of Solutions: Size of~$\solutionSpace$}%
\only<-9>{%
\begin{itemize}%
\item OK, we want to solve a JSSP instance%
\item<2-> How many possible candidate solutions are there?%
\item<3-> If we allow arbitrary useless waiting times between jobs, then we could create arbitrarily many different valid Gantt charts for any problem instance.%
\item<4-> Let us assume that no time is wasted by waiting unnecessarily -- which is what our search space representation does, too.%
\item<5-> If there was only 1~machine, then we would have $\jsspJobs!=1*2*3*4*5*\dots*\jsspJobs$ possible ways to arrange the $\jsspJobs$~jobs.%
\item<6-> If there are 2~machines, this gives us $(\jsspJobs!)*(\jsspJobs!)=(\jsspJobs!)^2$ choices.%
\item<7-> \only<7>{For three machines, we are at $(\jsspJobs!)^3$.}%
\only<8->{\alert<8>{For $\jsspMachines$~machines, we are at $(\jsspJobs!)^{\jsspMachines}$ possible solutions.}}%
\item<9-> \alert<9>{But some may be wrong, i.e., contain deadlocks!}%
\end{itemize}%
}%
%
\locate{10-}{%
\resizebox{0.65\linewidth}{!}{\begin{tabular}{lrrrr}%
\hline%
name&$\jsspJobs$&$\jsspMachines$&$\min(\#\text{feasible})$&$\left|\solutionSpace\right|$%
\\%
\hline%
&2&2&3&4\\%
\uncover<11->{&2&3&4&8\\%
\uncover<12->{&2&4&5&16\\%
\uncover<13->{&2&5&6&32\\%
\uncover<14->{&3&2&22&36\\%
&3&3&63&216\\%
&3&4&147&1'296\\%
&3&5&317&7'776\\%
\uncover<6->{&4&2&244&576\\%
&4&3&1'630&13'824\\%
&4&4&7'451&331'776\\%
\uncover<15->{\texttt{demo}&4&5&&7'962'624\\%
\texttt{la24}&15&10&&$\approx$~1.462*10\textsuperscript{121}\\%
\texttt{abz7}&20&15&&$\approx$~6.193*10\textsuperscript{275}\\%
\texttt{yn4}&20&20&&$\approx$~5.278*10\textsuperscript{367}\\%
\texttt{swv15}&50&10&&$\approx$~6.772*10\textsuperscript{644}\\%
\hline%
}}}}}}%
\end{tabular}}%
}{0.175}{0.124}%
%
\end{frame}%
%
\begin{frame}%
\frametitle{Size of Search Space $\searchSpace$}%
\only<-2,5->{%
\begin{itemize}%
\item Our search space~$\searchSpace$ is not the same as the solution space~$\solutionSpace$.
\item<2-> How many points are in our representations of the solution space?%
\item<5-> Both $\searchSpace$ and $\solutionSpace$ are very big for any relevant problem size.%
\item<6-> $\searchSpace$ is bigger, we pay with size for the simplicity and the avoidance of infeasible solutions.%
\end{itemize}%
}%
%
\locate{3}{%
\resizebox{0.65\linewidth}{!}{\begin{tabular}{lrrrrr}%
\hline%
name&$\jsspJobs$&$\jsspMachines$&$\left|\solutionSpace\right|$&$\left|\searchSpace\right|$
\\%
\hline%
&3&2&36&90\\%
&3&3&216&1'680\\%
&3&4&1'296&34'650\\%
&3&5&7'776&756'756\\%
&4&2&576&2'520\\%
&4&3&13'824&369'600\\%
&4&4&331'776&63'063'000\\%
&5&2&14'400&113'400\\%
&5&3&1'728'000&168'168'000\\%
&5&4&207'360'000&305'540'235'000\\%
&5&5&24'883'200'000&623'360'743'125'120\\%
\texttt{demo}&4&5&7'962'624&11'732'745'024\\%
\texttt{la24}&15&10&$\approx$~1.462*10\textsuperscript{121}&$\approx$~2.293*10\textsuperscript{164}\\%
\texttt{abz7}&20&15&$\approx$~6.193*10\textsuperscript{275}&$\approx$~1.432*10\textsuperscript{372}\\%
\texttt{yn4}&20&20&$\approx$~5.278*10\textsuperscript{367}&$\approx$~1.213*10\textsuperscript{501}\\%
\texttt{swv15}&50&10&$\approx$~6.772*10\textsuperscript{644}&$\approx$~1.254*10\textsuperscript{806}\\%
\hline%
\end{tabular}}%
}{0.175}{0.18}%
%
\locateGraphic{4}{width=0.77\paperwidth}{graphics/size/jssp_searchspace_size_scale}{0.115}{0.03}%
\end{frame}%
%
\section{Search Operators}%
%
\begin{frame}[t]%
\frametitle{Search Operators}%
\begin{itemize}%
\item Another general structure element needed by many optimization algorithms are \alert<1>{search operators}.%
\end{itemize}%
\uncover<2->{%
\begin{definition}{Search Operator}%
An \mbox{$k$-ary} \alert{search operator}~$\searchOp:\searchSpace^k\mapsto\searchSpace$ is a left-total relation which accepts $k$~points in the search space~$\searchSpace$ as input and returns one point in the search space as output.%
\end{definition}%
\uncover<3->{%
\only<-4,6,8,10->{%
\begin{itemize}%
\item Based on their arity~$k$, we can distinguish the following most common operator types:%
\uncover<4->{%
\begin{itemize}%
\item<4-> nullary operators ($k=0$) generate one (random) point in~$\searchSpace$.%
\item<6-> unary operators ($k=1$) take one point from~$\searchSpace$ as input and return another (similar) point.%
\item<8-> binary operators ($k=2$) take two points from~$\searchSpace$ as input and return another point which should be similar to both.%
\end{itemize}%
}%
\item<10-> We will discuss concrete implementations of the operators later.%
\end{itemize}%
}}}%
%
\only<5>{\listing{0.9}{0.95}{language=myJava,mathescape=false}{code/INullarySearchOperator.java}}%
\only<7>{\listing{0.9}{0.95}{language=myJava,mathescape=false}{code/IUnarySearchOperator.java}}%
\only<9>{\listing{0.9}{0.95}{language=myJava,mathescape=false}{code/IBinarySearchOperator.java}}%
%
\end{frame}%
%
%
\section{Termination}%
%
\begin{frame}%
\frametitle{Searching and Stopping}%
\begin{itemize}%
\item Eventually, we will have a program that uses the search operators efficiently to find elements in the set~$\searchSpace$ which correspond to good solutions in~$\solutionSpace$.%
\item<2-> How long should it run?%
\item<3-> When can it stop?%
\item<4-> This is called the \alert{termination criterion}.%
\end{itemize}%
\end{frame}%
%
\begin{frame}%
\frametitle{When to stop?}%
\begin{itemize}%
\item We assume that a human operator receives the job information, enters them into a computer (as JSSP instance), and then goes to drink a coffee.%
\item<2-> Can we solve larger, hard JSSPs with such \emph{huge} numbers of potential solutions until she comes back?%
\item<3-> Probably not.%
\item<4-> The best algorithms guaranteeing to find the optimal solution for our JSSPs may need a runtime growing exponential with $\jsspMachines$ and $\jsspJobs$.\cite{LLRKS1993SASAAC,CPW1998AROMSCAAA}
\item<5-> Even algorithms that just guarantee to be a constant factor worse than the optimum (like, 1\% longer, 10 times longer\dots) cannot faster on the JSSP in the worst case!\cite{WHHHLSS1997SSS,JMSO2005ASFJSSPWCPT,MS2011HOAFAJSSP}
\item<6-> So?\uncover<7->{ \dots\ The operator drinks a coffee.\uncover<8->{ \dots\ We have a about three minutes.\uncover<9->{ \dots\ Let's look for the algorithm implementation that can give us the best solution quality within that time window.}}}%
\item<10-> This is the termination criterion we will use on our JSSP example problem in this lecture.%
\item<11-> Obviously, in other scenarios, there might be vastly different criteria\dots%
\end{itemize}%
\end{frame}%
%
\section{Summary}%
%
\begin{frame}%
\frametitle{Summary}%
\begin{itemize}%
\item This was the most complicated lesson in this course!%
\item<2-> Thank you for sticking with me during this.%
\item<3-> What we have learned is the most basic process when attacking any optimization problem\only<-3>{!}\uncover<4->{:%
\begin{enumerate}%
\item Understand how the scenario / input data is defined!%
\item<5-> Make a data structure~$\solutionSpace$ for the solutions, which can contain all the information that the end user needs and considers as a full solution to the problem!%
\item<6-> Define the objective function~$\objf$, which rates how good a solution is!%
\item<7-> Is $\solutionSpace$ easy to understand and to process by an algorithm?\uncover<8->{ If \textcolor{green}{yes}: cool.\uncover<9->{ If \textcolor{red}{no}: define a simple data structure~$\searchSpace$ and a translation~$\repMap$ from~$\searchSpace$ to~$\solutionSpace$!}}%
\item<10-> Understand when we need to stop the search!%
\end{enumerate}%
}%
\item<11-> If we have this, we can directly use most of the algorithms in the rest of the lecture (almost) as-is.%
\end{itemize}%
\end{frame}%
%
\begin{frame}%
\frametitle{Summary}%
\begin{itemize}%
\item We now have the basic tools to search and find solutions for the JSSP.%
\item<2-> Many other problems are similar and can be represented in a similar way.%
\item<3-> The key is often to translate the complicated task to work with a complex data structure~$\solutionSpace$ (e.g., Gantt diagram with many constraints) to a simpler scenario where I only need to deal with a basic data structure~$\searchSpace$ (like a list of integer numbers with few constraints) by putting the ``complicated'' rules into a mapping $\repMap:\searchSpace\mapsto\solutionSpace$.%
\item<4-> If I can do that, then from now on I do not need to worry about~$\solutionSpace$ and its rules any more -- I only need to work with~$\searchSpace$, which is easier to understand and to program.%
\item<5-> Let us now try to solve the JSSP using metaheuristics that search inside~$\searchSpace$ (and thus can find solutions in~$\solutionSpace$\only<6->{ within 3 minutes}).%
\end{itemize}%
\end{frame}%
%
%
\endPresentation%
\end{document}%%
\endinput%
%