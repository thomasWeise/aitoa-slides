\documentclass[mathserif]{beamer}%
%
\xdef\sharedDir{../_shared_}%
\RequirePackage{\sharedDir/styles/slides}%
%
\subtitle{2. Structure}%
%
\begin{document}%
%
\startPresentation{}%
%
%
\section{Introduction}%
%
\begin{frame}%
\frametitle{The Structure of Optimization}%
\begin{itemize}%
\item So we know roughly what an optimization problem is and that metaheuristics\cite{aitoa,WGOEB,GK2003HOM,MF2004HTSIMH} are algorithms to solve them.%
\item<2-> But we do not really know yet how that works.%
\item<3-> We will approach this topic based on an example from the field of Smart Manufacturing.
\item<4-> We will first learn about the basic ingredients that make up an optimization task.%
\item<5-> Then we will step-by-step work our way from stupid to good metaheuristics for solving it.%
\end{itemize}%
\end{frame}%
%
%
\begin{frame}[t]%
\frametitle{Components of an Optimization Problem}%
\begin{itemize}%
\item From the perspective of a programmer, we can say that an optimization problem has the following components\uncover<2->{:%
\begin{enumerate}%
\item the input data which specifies the problem instance~$\instance$ to be solved\only<3>{ -- we develop software for solving a class of problems, but this software is applied to specific problem instances, the actual scenarios}%
\item<4-> a data type~$\solutionSpace$ for the candidate solutions~$\solspel\in\solutionSpace$\uncover<2->{, and}%
\item<5-> an objective function~$\objf:\solutionSpace\mapsto\realNumbers$\only<5>{, which rates ``how good'' a candidate solution~$\solspel\in\solutionSpace$ is}.%
\end{enumerate}%
}%
\item<6-> Usually, in order to \emph{practically implement} an optimization approach, there also will be%
\uncover<7->{%
\begin{enumerate}\setcounter{enumi}{3}%
\item a search space~$\searchSpace$\only<7>{, i.e., a simpler data structure for internal use, which can more efficiently be processed by an optimization algorithm than~$\solutionSpace$}\uncover<8->{,}%
\item<8-> a representation mapping~$\repMap:\searchSpace\mapsto\solutionSpace$\only<8>{, which translates ``points''~$\sespel\in\searchSpace$ to candidate solutions~$\solspel\in\solutionSpace$}\uncover<9->{,}%
\item<9-> search operators~$\searchOp:\searchSpace^n\mapsto\searchSpace$\only<9>{, which allow for the iterative exploration of the search space~$\searchSpace$}\uncover<10->{, and}%
\item<10-> a termination criterion\only<10>{, which tells the optimization process when to stop}.%
\end{enumerate}%
}%
\item<11-> Looks complicated\only<12->{, but don't worry}.\only<13->{ We will do this one-by-one}.%
\item<14-> We want to get an understanding of the structure of optimization problems from the metaheuristic perspective by looking at one concrete problem from production planning.%
\end{itemize}%
%
\end{frame}%
%
\section{Example Problem: Job Shop Scheduling}%
%
\begin{frame}%
\frametitle{Job Shop Problem}%
\locateGraphic{1}{width=0.9\paperwidth}{graphics/jssp/jssp_sketch_01}{0.05}{0.2}%
\locateGraphic{2}{width=0.9\paperwidth}{graphics/jssp/jssp_sketch_02}{0.05}{0.2}%
\locateGraphic{3}{width=0.9\paperwidth}{graphics/jssp/jssp_sketch_03}{0.05}{0.2}%
\locateGraphic{4}{width=0.9\paperwidth}{graphics/jssp/jssp_sketch_04}{0.05}{0.2}%
\locateGraphic{5}{width=0.9\paperwidth}{graphics/jssp/jssp_sketch_05}{0.05}{0.2}%
\locateGraphic{6}{width=0.9\paperwidth}{graphics/jssp/jssp_sketch_06}{0.05}{0.2}%
\locateGraphic{7}{width=0.9\paperwidth}{graphics/jssp/jssp_sketch_07}{0.05}{0.2}%
\locateGraphic{8}{width=0.9\paperwidth}{graphics/jssp/jssp_sketch_08}{0.05}{0.2}%
\locateGraphic{9}{width=0.9\paperwidth}{graphics/jssp/jssp_sketch_09}{0.05}{0.2}%
\locateGraphic{10}{width=0.9\paperwidth}{graphics/jssp/jssp_sketch_10}{0.05}{0.2}%
\locateGraphic{11}{width=0.9\paperwidth}{graphics/jssp/jssp_sketch_11}{0.05}{0.2}%
\locateGraphic{12}{width=0.9\paperwidth}{graphics/jssp/jssp_sketch_12}{0.05}{0.2}%
\locateGraphic{13}{width=0.9\paperwidth}{graphics/jssp/jssp_sketch_13}{0.05}{0.2}%
\end{frame}%
%
\begin{frame}%
\frametitle{Job Shop Scheduling Problem}%
\begin{itemize}%
\item The Job Shop Scheduling Problem (JSSP)\cite{GLLRK1979OAAIDSASAS,LLRKS1993SASAAC,L1982RRITTOMS,T199BFBSP,BDP1996TJSSPCANST} is a classical optimization problem.%
\item<2-> We have a factory with~$\jsspMachines$ machines.%
\item<3-> We need to fulfill~$\jsspJobs$ production requests, the \emph{jobs}.%
\item<4-> Each job will need to be processed by some or all of the machines in a job-specific order.%
\item<5-> Also, each job will require a job-specific time at a given machine.%
\item<6-> The goal is to fulfill all tasks as quickly as possible.%
\item<7-> This scenario also encompasses simpler problems, e.g., where all jobs ``are the same.''%
\item<8-> This problem is \mbox{$\NPprefix$-hard}.\cite{K1972RACP,C1971TCOTPP}%
\end{itemize}%
\end{frame}%
%
\begin{frame}%
\frametitle{What we will do}%
\begin{itemize}%
\item In this course, we will use the JSSP as example domain.%
\item<2-> We will discuss all components of an optimization problem based on this example.%
\item<3-> We will discuss several different optimization algorithms -- and apply them to this problem.%
\item<4-> \alert{But}\uncover<5->{~we will do this from an \emph{educational} perspective}%
\item<5-> We will \emph{not} focus on the best possible data structures or highest possible efficiency.%
\item<6-> It needs years of research to get there\dots%
\item<7-> We will, instead, approach the JSSP in the same way you would approach a completely new problem domain\uncover<8->{: develop a working approach\uncover<9->{, test and compare different working approaches\uncover<10->{, (normally you would then improve them further, but we will skip this)}}}%
\end{itemize}%
\end{frame}%
%
\section{Problem Instance}%
%
\begin{frame}%
\frametitle{The Input: Problem Instances}%
\begin{itemize}%
\item The JSSP is a \alert<2>{type} of problem.%
\item<2-> A concrete scenario, with a specific number of machines and with specific jobs, is called an \mbox{\alert<2>{instance~$\jsspInstance$}}.%
\item<3-> It is common in research that there collections of instances for a given problem, so that we can test algorithms and compare their performance (of course, you can only compare results if they are for the same scenario).%
\item<4-> \mbox{Beasley\cite{B1990OLDTPBEM}} manages the \emph{OR~Library} of benchmark datasets from different fields of operations research (OR)%
\item<5-> He also provides several example instances of the JSSP at \url{http://people.brunel.ac.uk/~mastjjb/jeb/orlib/jobshopinfo.html}.%
\item<6-> More information about these instances has been collected by \mbox{van Hoorn\cite{vH2015JSIAS,vH2018TCSOBOBIOTJSSP}} at \url{http://jobshop.jjvh.nl}.
\item<7-> What do such JSSP instances look like?%
\end{itemize}%
\end{frame}%
%
\begin{frame}[b]%
\frametitle{Demo Instance}%
%
\definecolor{c1}{RGB}{165,45,132}%
\definecolor{c2}{RGB}{244,114,22}%
\definecolor{c3}{RGB}{46,49,146}%
\definecolor{c4}{RGB}{99,194,13}%
\definecolor{c5}{RGB}{236,0,140}%
%
\locateGraphic{1}{width=0.98\paperwidth}{graphics/instance/demo_instance_clear}{0.01}{0.19}%
\locateGraphic{2}{width=0.98\paperwidth}{graphics/instance/demo_instance_jobs}{0.01}{0.19}%
\locateGraphic{3}{width=0.98\paperwidth}{graphics/instance/demo_instance_machines}{0.01}{0.19}%
\locateGraphic{4,8}{width=0.98\paperwidth}{graphics/instance/demo_instance_job_1}{0.01}{0.19}%
\locateGraphic{5}{width=0.98\paperwidth}{graphics/instance/demo_instance_job_2}{0.01}{0.19}%
\locateGraphic{6}{width=0.98\paperwidth}{graphics/instance/demo_instance_job_3}{0.01}{0.19}%
\locateGraphic{7}{width=0.98\paperwidth}{graphics/instance/demo_instance_job_4}{0.01}{0.19}%
%
\locateGraphic{9}{width=0.98\paperwidth}{graphics/instance/demo_instance_job_1_1}{0.01}{0.19}%
\locateGraphic{10}{width=0.98\paperwidth}{graphics/instance/demo_instance_job_1_2}{0.01}{0.19}%
\locateGraphic{11}{width=0.98\paperwidth}{graphics/instance/demo_instance_job_1_3}{0.01}{0.19}%
\locateGraphic{12}{width=0.98\paperwidth}{graphics/instance/demo_instance_job_1_4}{0.01}{0.19}%
\locateGraphic{13}{width=0.98\paperwidth}{graphics/instance/demo_instance_job_1_5}{0.01}{0.19}%
%
\locateGraphic{14}{width=0.98\paperwidth}{graphics/instance/demo_instance_job_2_1}{0.01}{0.19}%
\locateGraphic{15}{width=0.98\paperwidth}{graphics/instance/demo_instance_job_2_2}{0.01}{0.19}%
\locateGraphic{16}{width=0.98\paperwidth}{graphics/instance/demo_instance_job_2_3}{0.01}{0.19}%
\locateGraphic{17}{width=0.98\paperwidth}{graphics/instance/demo_instance_job_2_4}{0.01}{0.19}%
\locateGraphic{18}{width=0.98\paperwidth}{graphics/instance/demo_instance_job_2_5}{0.01}{0.19}%
%
\locateGraphic{19}{width=0.98\paperwidth}{graphics/instance/demo_instance_job_3_5}{0.01}{0.19}%
\locateGraphic{20}{width=0.98\paperwidth}{graphics/instance/demo_instance_job_4_5}{0.01}{0.19}%
%
\locateGraphic{21}{width=0.98\paperwidth}{graphics/instance/demo_instance}{0.01}{0.19}%
%
\only<9-13>{%
Job 0 first needs to be processed by \textcolor{c1}{machine 0 for 10 time units}%
\uncover<10->{, it then goes to \textcolor{c2}{machine 1 for 20 time units}%
\uncover<11->{, it then goes to \textcolor{c3}{machine 2 for 20 time units}%
\uncover<12->{, it then goes to \textcolor{c4}{machine 3 for 40 time units}%
\uncover<13->{, and finally it goes to \textcolor{c5}{machine 4 for 10 time units}.%
}}}}}%
%
\only<14-18>{%
Similarly, Job 1 first needs to be processed by \textcolor{c1}{machine 1 for 20 time units}%
\uncover<15->{, it then goes to \textcolor{c2}{machine 0 for 10 time units}%
\uncover<16->{, it then goes to \textcolor{c3}{machine 3 for 30 time units}%
\uncover<17->{, it then goes to \textcolor{c4}{machine 2 for 50 time units}%
\uncover<18->{, and finally it goes to \textcolor{c5}{machine 4 for 30 time units}.%
}}}}}%
\only<19>{%
Job 2 first needs to be processed by \textcolor{c1}{machine 2 for 30 time units}%
, it then goes to \textcolor{c2}{machine 1 for 20 time units}%
, it then goes to \textcolor{c3}{machine 4 for 12 time units}%
, it then goes to \textcolor{c4}{machine 3 for 40 time units}%
, and finally it goes to \textcolor{c5}{machine 0 for 10 time units}.%
}%
\only<20>{%
And Job 3 first needs to be processed by \textcolor{c1}{machine 4 for 50 time units}%
, it then goes to \textcolor{c2}{machine 3 for 30 time units}%
, it then goes to \textcolor{c3}{machine 2 for 15 time units}%
, it then goes to \textcolor{c4}{machine 0 for 20 time units}%
, and finally it goes to \textcolor{c5}{machine 1 for 15 time units}.%
}%
\strut\\\strut\medskip\strut%
%
\end{frame}%
%
\begin{frame}[t]%
\frametitle{Instance \texttt{abz7}}%
Instance \texttt{abz7} by Adams et~al.\cite{ABZ1988TSBPFJSS}%
\locateGraphic{}{width=0.95\paperwidth}{graphics/instance/abz7}{0.02}{0.25}%
\end{frame}%
%
\begin{frame}[t]%
\frametitle{Instance \texttt{la24}}%
Instance \texttt{la24} by Lawrence\cite{L1998RCPSAEIOHSTS}.%
\locateGraphic{}{width=0.9\paperwidth}{graphics/instance/la24}{0.05}{0.275}%
\end{frame}%
%
\begin{frame}[t]%
\frametitle{Instance \texttt{swv15}}%
Instance \texttt{swv15} by Storer et~al.\cite{SWV1992NSSFSPWATJSS}%
\locateGraphic{}{width=0.45\paperwidth}{graphics/instance/swv15}{0.4}{0.2}%
\end{frame}%
%
\begin{frame}[t]%
\frametitle{Instance \texttt{yn4}}%
Instance \texttt{yn4} by Yamada and Nakano\cite{YN1992AGAATLSJSI}.%
\locateGraphic{}{width=0.95\paperwidth}{graphics/instance/yn4}{0.02}{0.25}%
\end{frame}%
%
\begin{frame}[containsverbatim,fragile]%
\frametitle{Problem Instance Data in Java}%
\begin{itemize}%
\item How can we represent such data in Java program code?%
\end{itemize}%
\uncover<2->{%
\listing{0.85}{0.95}{language=myJava,mathescape=false}{code/JSSPInstance.java}%
}%
\end{frame}
%
%
\section{Solution Space}%
%
\begin{frame}[t]%
\frametitle{Output: Candidate Solutions and Solution Space~$\solutionSpace$}%
\only<-4,9->{%
\begin{itemize}%
\item We now know how a problem instance of the JSSP looks like, i.e., the \alert{input} we get.%
\item<2-> But what \alert{output} should we produce?%
\item<3-> In other words, what is a solution for an instance of the JSSP?%
\item<4-> Basically, a Gantt Chart\cite{W2003GCACA,K2000SORCP}.%
\item<9-> A Gantt chart is a diagram which assigns each sub-job on each machine a start and end time.%
\item<10-> The solution space~$\solutionSpace$ is the set of all possible feasible solutions for one JSSP instance.%
\item<11-> One possible solution is called \alert{candidate solution} and it can be illustrated as Gantt chart.%
\end{itemize}%
}%
\only<5-8>{%
\begin{center}
\medskip%
\only<-5>{\large{one possible solution for the \texttt{demo} instance}, illustrated as Gantt chart}%
\only<6>{\large{one possible solution for the \texttt{la24} instance}, illustrated as Gantt chart}%
\only<7>{\large{one possible solution for the \texttt{yn4} instance}, illustrated as Gantt chart}%
\only<8->{\large{one possible solution for the \texttt{swv15} instance}, illustrated as Gantt chart}%
\end{center}%
}%
\locateGraphic{5}{width=0.9\paperwidth}{graphics/gantt/gantt_demo}{0.05}{0.286}%
\locateGraphic{6}{width=0.9\paperwidth}{graphics/gantt/gantt_la24}{0.05}{0.286}%
\locateGraphic{7}{width=0.9\paperwidth}{graphics/gantt/gantt_yn4}{0.05}{0.286}%
\locateGraphic{8}{width=0.9\paperwidth}{graphics/gantt/gantt_swv15}{0.05}{0.286}%
\end{frame}%
%
\begin{frame}[t]%
\frametitle{As Java Class}%
\only<-3>{%
\begin{itemize}%
\item We now need to represent this information as a Java class.%
\item<3-> Each of the~$\jsspMachines$ \jcodeil{int[]} lists in \codeil{schedule} holds~$\jsspJobs$ operations for each machine as three values jobID, start time, end time, i.e., has length~$3\jsspJobs$.%
\end{itemize}%
\uncover<2-3>{%
\listing{0.85}{0.95}{language=myJava,mathescape=false}{code/JSSPCandidateSolution.java}%
}%
}%
%
\locateGraphic{4}{width=0.8\paperwidth}{graphics/candidate_solution/demo_candidate_solution}{0.1}{0.1}%
\locateGraphic{5}{width=0.8\paperwidth}{graphics/candidate_solution/demo_candidate_solution_machines_colored}{0.1}{0.1}%
\locateGraphic{6}{width=0.8\paperwidth}{graphics/candidate_solution/demo_candidate_solution_0_01}{0.1}{0.1}%
\locateGraphic{7}{width=0.8\paperwidth}{graphics/candidate_solution/demo_candidate_solution_0_02}{0.1}{0.1}%
\locateGraphic{8}{width=0.8\paperwidth}{graphics/candidate_solution/demo_candidate_solution_0_03}{0.1}{0.1}%
\locateGraphic{9}{width=0.8\paperwidth}{graphics/candidate_solution/demo_candidate_solution_0_04}{0.1}{0.1}%
\locateGraphic{10}{width=0.8\paperwidth}{graphics/candidate_solution/demo_candidate_solution_0_05}{0.1}{0.1}%
\locateGraphic{11}{width=0.8\paperwidth}{graphics/candidate_solution/demo_candidate_solution_0_06}{0.1}{0.1}%
\locateGraphic{12}{width=0.8\paperwidth}{graphics/candidate_solution/demo_candidate_solution_0_07}{0.1}{0.1}%
\locateGraphic{13}{width=0.8\paperwidth}{graphics/candidate_solution/demo_candidate_solution_0_08}{0.1}{0.1}%
\locateGraphic{14}{width=0.8\paperwidth}{graphics/candidate_solution/demo_candidate_solution_0_09}{0.1}{0.1}%
\locateGraphic{15}{width=0.8\paperwidth}{graphics/candidate_solution/demo_candidate_solution_0_10}{0.1}{0.1}%
\locateGraphic{16}{width=0.8\paperwidth}{graphics/candidate_solution/demo_candidate_solution_0_11}{0.1}{0.1}%
\locateGraphic{17}{width=0.8\paperwidth}{graphics/candidate_solution/demo_candidate_solution_0_12}{0.1}{0.1}%
\locateGraphic{18}{width=0.8\paperwidth}{graphics/candidate_solution/demo_candidate_solution_1_01}{0.1}{0.1}%
\locateGraphic{19}{width=0.8\paperwidth}{graphics/candidate_solution/demo_candidate_solution_1_02}{0.1}{0.1}%
\locateGraphic{20}{width=0.8\paperwidth}{graphics/candidate_solution/demo_candidate_solution_1_03}{0.1}{0.1}%
\locateGraphic{21}{width=0.8\paperwidth}{graphics/candidate_solution/demo_candidate_solution_1_04}{0.1}{0.1}%
\locateGraphic{22}{width=0.8\paperwidth}{graphics/candidate_solution/demo_candidate_solution_1_05}{0.1}{0.1}%
\locateGraphic{23}{width=0.8\paperwidth}{graphics/candidate_solution/demo_candidate_solution_1_06}{0.1}{0.1}%
\locateGraphic{24}{width=0.8\paperwidth}{graphics/candidate_solution/demo_candidate_solution_1_07}{0.1}{0.1}%
\locateGraphic{25}{width=0.8\paperwidth}{graphics/candidate_solution/demo_candidate_solution_1_08}{0.1}{0.1}%
\locateGraphic{26}{width=0.8\paperwidth}{graphics/candidate_solution/demo_candidate_solution_1_09}{0.1}{0.1}%
\locateGraphic{27}{width=0.8\paperwidth}{graphics/candidate_solution/demo_candidate_solution_1_10}{0.1}{0.1}%
\locateGraphic{28}{width=0.8\paperwidth}{graphics/candidate_solution/demo_candidate_solution_1_11}{0.1}{0.1}%
\locateGraphic{29}{width=0.8\paperwidth}{graphics/candidate_solution/demo_candidate_solution_1_12}{0.1}{0.1}%
\end{frame}%
%
\section{Objective Function}%
%
\begin{frame}[t]%
\frametitle{Solution Quality}%
\begin{itemize}%
\only<-6,17-18>{%
\item So we have identified what the possible solutions to our problems are and know how to store them in a data structure.%
\item<2-> How do we rate the quality of a solution?%
\item<3-> A Gantt chart $\solspel_{1}\in\solutionSpace$ is a better solution to our problem than another chart $\solspel_{2}\in\solutionSpace$ if \textcolor<18>{red}{it allows us to complete our work faster}.%
}%
\only<-18>{%
\item<4-> The \alert<4>{objective function} $\objf:\solutionSpace\mapsto\realNumbers$ is the \emph{makespan}\uncover<5->{, the time when the last sub-job is completed\uncover<6->{, the right-most edge of any bar in the Gantt chart.}}%
\item<17-> This objective function is subject to minimization: smaller values are better.%
}%
\item<18-> \alert{A Gantt chart $\solspel_{1}\in\solutionSpace$ is a better solution to our problem than another chart $\solspel_{2}\in\solutionSpace$ if \textcolor<18>{red}{$\objf(\solspel_{1})<\objf(\solspel_{2})$}.}%
\end{itemize}%
%
\locateGraphic{7}{width=0.9\paperwidth}{graphics/makespan/no_makespan_demo}{0.05}{0.3}%
\locateGraphic{8}{width=0.9\paperwidth}{graphics/makespan/makespan_demo}{0.05}{0.3}%
\locateGraphic{9}{width=0.9\paperwidth}{graphics/makespan/no_makespan_abz7}{0.05}{0.3}%
\locateGraphic{10}{width=0.9\paperwidth}{graphics/makespan/makespan_abz7}{0.05}{0.3}%
\locateGraphic{11}{width=0.9\paperwidth}{graphics/makespan/no_makespan_la24}{0.05}{0.3}%
\locateGraphic{12}{width=0.9\paperwidth}{graphics/makespan/makespan_la24}{0.05}{0.3}%
\locateGraphic{13}{width=0.9\paperwidth}{graphics/makespan/no_makespan_swv15}{0.05}{0.3}%
\locateGraphic{14}{width=0.9\paperwidth}{graphics/makespan/makespan_swv15}{0.05}{0.3}%
\locateGraphic{15}{width=0.9\paperwidth}{graphics/makespan/no_makespan_yn4}{0.05}{0.3}%
\locateGraphic{16}{width=0.9\paperwidth}{graphics/makespan/makespan_yn4}{0.05}{0.3}%
%
%
\locateGraphic{19}{width=0.5\paperwidth}{graphics/makespan/makespan_good_abz7}{0.15}{0.25}%
\locateGraphic{19}{width=0.5\paperwidth}{graphics/makespan/makespan_bad_abz7}{0.35}{0.6}%
\locateGraphic{20}{width=0.9\paperwidth}{graphics/makespan/makespan_good_abz7}{0.05}{0.3}%
%
\locateGraphic{21}{width=0.5\paperwidth}{graphics/makespan/makespan_bad_la24}{0.15}{0.25}%
\locateGraphic{21}{width=0.5\paperwidth}{graphics/makespan/makespan_good_la24}{0.35}{0.6}%
\locateGraphic{22}{width=0.9\paperwidth}{graphics/makespan/makespan_good_yn4}{0.05}{0.3}%
%
\locateGraphic{23}{width=0.5\paperwidth}{graphics/makespan/makespan_bad_swv15}{0.15}{0.25}%
\locateGraphic{23}{width=0.5\paperwidth}{graphics/makespan/makespan_good_swv15}{0.35}{0.6}%
\locateGraphic{24}{width=0.9\paperwidth}{graphics/makespan/makespan_good_swv15}{0.05}{0.3}%
%
\locateGraphic{25}{width=0.5\paperwidth}{graphics/makespan/makespan_good_yn4}{0.15}{0.25}%
\locateGraphic{25}{width=0.5\paperwidth}{graphics/makespan/makespan_bad_yn4}{0.35}{0.6}%
\locateGraphic{26}{width=0.9\paperwidth}{graphics/makespan/makespan_good_yn4}{0.05}{0.3}%
%
\end{frame}%
%
\endPresentation%
\end{document}%%
\endinput%
%
